\documentclass[../../projlab]{subfiles}
\begin{document}

\ifSubfilesClassLoaded{
	\coverpage{1. Leadás}
}{}



\chapter{Követelmény, projekt, funkcionalitás}

\section{Bevezetés}

\markdownInput[shiftHeadings=2]{\subfix{introduction.md}}


%\subsection{Hivatkozások}
%\urlref{http://iit.bme.hu/}{BME IIT - Programozás alapjai 3. segédanyagok, Szoftvertechnikák segédanyagok, Szoftver projekt laboratórium feladatok}

\section{Áttekintés}
\markdownInput[shiftHeadings=2]{\subfix{overview.md}}


\section{Követelmények}
\subsection{Funkcionális követelmények}


\begin{funkovetelmeny}
	%Azonosító
	{A01}
	%Prioritás
	{Magas}
	%Forrás
	{Feladat}
	%Use case
	{Action}
	%Ellenőrzés
	{Bemutatás}
	%Leírás
	{A játékos egy vagy több telepest irányít.}
\end{funkovetelmeny}

\begin{funkovetelmeny}
	{A02}
	{Magas}
	{Feladat}
	{Action}
	{}
	{A játék köralapú, egyszerre egy telepes egy lépést tud elvégezni, a lépést a játékos dönti el.}
\end{funkovetelmeny}

\begin{funkovetelmeny}
	{A03}
	{Magas}
	{Feladat}
	{Move}
	{}
	{Egy telepes egyszerre egy aszteroidán lehet, és egy lépésként egy szomszédos aszteroidára utazhat.}
    
\end{funkovetelmeny}

\begin{funkovetelmeny}
	{A04}
	{Magas}
	{Feladat}
	{Drill, Mine, Move asteroid belt}
	{}
	{Minden aszteroidának van köpenye és magja. A mag lehet üres, vagy egy fajta nyersanyagot (vas, szén, urán stb.) tartalmaz, amelyek között lehet radioaktív is.}
    
\end{funkovetelmeny}

\begin{funkovetelmeny}
	{A05}
	{Magas}
	{Feladat}
	{Drill}
	{}
	{Ha egy aszteroida köpenye még nincs teljesen átfúrva, akkor a telepes egy lépésként egy egységnyivel mélyítheti a köpenyen lévő lyukat.}
    
\end{funkovetelmeny}

\begin{funkovetelmeny}
	{A06}
	{Magas}
	{Feladat}
	{Mine}
	{}
	{Ha egy aszteroida köpenye teljesen át van fúrva, és a magja nem üres, akkor a telepes bányászhat az adott aszteroidán.}
    
\end{funkovetelmeny}

\begin{funkovetelmeny}
	{A07}
	{Magas}
	{Feladat}
	{Mine}
	{}
	{A bányászáskor a telepes egy lépésként meg tud szerezni az aszteroida magjában található erőforrásból egy egységnyit. }
    
\end{funkovetelmeny}

\begin{funkovetelmeny}
	{A08}
	{Magas}
	{Feladat}
	{Mine}
	{}
	{Egy telepesnél maximum 10 egységnyi nyersanyag lehet. }
    
\end{funkovetelmeny}

\begin{funkovetelmeny}
	{A09}
	{Magas}
	{Feladat}
	{Place resource}
	{}
	{Ha éppen az aszteroida magja üres, akkor a telepes egy lépésként visszahelyezhet egy egységnyi már kinyert nyersanyagot.}
    
\end{funkovetelmeny}

\begin{funkovetelmeny}
	{A10}
	{Magas}
	{Feladat}
	{Build robot}
	{}
	{A telepes egy egységnyi vas, egy egységnyi szén és egy egységnyi urán felhasználással egy lépésként készíthet egy autonóm robotot.}
    
\end{funkovetelmeny}

\begin{funkovetelmeny}
	{A11}
	{Magas}
	{Feladat}
	{Control robots}
	{}
	{A robot képes magától aszteroidák között mozogni, nem tud bányászni, de tud fúrni.}
    
\end{funkovetelmeny}

\begin{funkovetelmeny}
	{A12}
	{Magas}
	{Feladat}
	{Build gate, Place gate}
	{}
	{A telepes két egységnyi vas, egy egységnyi vízjég és egy egységnyi urán felhasználással egy lépésként tud készíteni egy teleportkapu-párt, majd egy lépésként lehelyezhet egy kaput egy aszteroidán. Egy telepes maximum 2 kaput tarthat magánál.}
    
\end{funkovetelmeny}

\begin{funkovetelmeny}
	{A13}
	{Magas}
	{Feladat}
	{Use gate, Place gate}
	{}
	{Egy teleportkapunak párja állandó, csak egy kapu lehelyezésével a kapu nem működik.}
    
\end{funkovetelmeny}

\begin{funkovetelmeny}
	{A14}
	{Magas}
	{Feladat}
	{Use gate, Control robots}
	{}
	{Bármely telepes és robot használhatja a kaput, a rajta lévő aszteroidától egy lépésként beléphet a kapuba, majd kijön a másik oldalon.}
    
\end{funkovetelmeny}

\begin{funkovetelmeny}
	{A15}
	{Magas}
	{Feladat}
	{Mine, Control robots}
	{}
	{Amennyiben napközelben egy olyan uránmagú aszteroida van, aminek a kérge teljesen át lett fúrva, akkor a mag felhevül és felrobban az aszteroidával együtt. Az aszteridán lévő telepesek meghalnak, a robotok szomszédos aszteroidákra kerülnek.}
    
\end{funkovetelmeny}

\begin{funkovetelmeny}
	{A16}
	{Magas}
	{Feladat}
	{Solar flare}
	{}
	{Változó időközönként napvihar keletkezik és egyszerre éri el az összes aszteroidát.}
    
\end{funkovetelmeny}

\begin{funkovetelmeny}
	{A17}
	{Magas}
	{Feladat}
	{Solar flare, Hide}
	{}
	{Amikor jön a napvihar, a telepes csak akkor élheti túl, ha megbújik egy üreges aszteroida magjában. }
    
\end{funkovetelmeny}

\begin{funkovetelmeny}
	{A18}
	{Magas}
	{Feladat}
	{Solar flare, Control robots}
	{}
	{A robot a napvihar közeledésekor magától megbújik a legközelebbi üreges aszteroida magjában aminek a kérge át van fúrva. Ha nem tud elbújni, akkor a robot megsemmisül. }
    Még meg kell beszélni
\end{funkovetelmeny}

\begin{funkovetelmeny}
	{A19}
	{Magas}
	{Feladat}
	{Build station}
	{}
	{Ha a telepesek összegyűjtenek mindegyik nyersanyagból legalább 3 egységet, és egy aszteroidában gyűjtik össze, akkor egy lépésként készíthetnek egy bázist.}
    
\end{funkovetelmeny}

\begin{funkovetelmeny}
	{A20}
	{Magas}
	{Feladat}
	{Check win condition}
	{}
	{A bázis elkészítéskor a játékosok megnyerik a játékot.}
    A játékosok választhatják azt hogy tovább játszanak.
\end{funkovetelmeny}

\begin{funkovetelmeny}
	{A21}
	{Magas}
	{Feladat}
	{Check win condition}
	{}
	{Ha minden telepes meghal, akkor a játékosok veszítenek.}
    
\end{funkovetelmeny}

\begin{funkovetelmeny}
	{B01}
	{Közepes}
	{Feladat}
	{-}
	{}
	{Több játékos tud együtt játszani egy közös csoportban.}
    
\end{funkovetelmeny}

\begin{funkovetelmeny}
	{B02}
	{Közepes}
	{Feladat}
	{-}
	{}
	{A játék indításkor a telepesek egymáshoz közeli helyen kezdenek, a köztük levő távolság maximum 5 lépés.}
    
\end{funkovetelmeny}

\begin{funkovetelmeny}
	{B03}
	{Közepes}
	{Feladat}
	{Move}
	{}
	{Egy aszteroidán bármennyi telepes vagy robot állhat és dolgozhat.}
    
\end{funkovetelmeny}

\begin{funkovetelmeny}
	{B04}
	{Közepes}
	{Feladat}
	{Place resource}
	{}
	{Minden aszteroida magjának lesz egy bizonyos nagysága, egy adott aszteroidában annyi nyersanyagot lehet tárolni, amennyi a magjába belefél.}
    
\end{funkovetelmeny}

\begin{funkovetelmeny}
	{B05}
	{Közepes}
	{Feladat}
	{Direct robot}
	{}
	{A telepes bármikor irányíthat a saját készített robotokat, vagy feltétlenül vándor a űrben, vagy mindig kísérel a tulajdonosát. }
    
\end{funkovetelmeny}

\begin{funkovetelmeny}
	{B06}
	{Közepes}
	{Feladat}
	{Direct robot}
	{}
	{Amikor egy robot kísérel a telepeset és éppen nem fúr, akkor mindig odamegy, ahol most a tulajdonosa feláll. Amikor megérkezi, akkor együtt fog dolgozni a telepessel, ha rajta lévő aszteroida nincs teljesen átfúrva, másképpen pedig utazik egy random szomszédosra.}
    
\end{funkovetelmeny}

\begin{funkovetelmeny}
	{B07}
	{Közepes}
	{Feladat}
	{Direct robot}
	{}
	{Fenti esetén amikor a robot megérkezik a szomszédos aszteroidán, vagy kezdődik fúrni, ha ez az aszteroida nincs teljesen átfúrva, vagy visszamegy a telepes meglátogat aszteroidára, és megismétli a fenti mozdulást.   }
    
\end{funkovetelmeny}

\begin{funkovetelmeny}
	{B08}
	{Közepes}
	{Csapat}
	{Place gate}
	{}
	{A teleportkapu lehelyezése után nem lehet mozgatni vagy lebontani, kivéve ha az aszteroida felrobban, és ilyenkor a megszűnt kapu párja is felrobban.}
    
\end{funkovetelmeny}

\begin{funkovetelmeny}
	{B09}
	{Közepes}
	{Feladat}
	{Move asteroid belt}
	{}
	{Az aszteroida mező mozog a Nap körül. A Naptól való távolsága folyamatosan változik.}
    
\end{funkovetelmeny}

\begin{funkovetelmeny}
	{B10}
	{Közepes}
	{Feladat}
	{View map}
	{}
	{A játékosok megtekinthetik mennyire vannak messze a Naptól.}
    
\end{funkovetelmeny}

\begin{funkovetelmeny}
	{B11}
	{Közepes}
	{Feladat}
	{Hide}
	{}
	{A napvihar érkezéskor egy üreges aszteroidában csak egy telepes tud elbújni, de akárhány robot is elbújhat a telepessel együtt ugyanabban az aszteroidában.}
    
\end{funkovetelmeny}

\subsection{Erőforrásokkal kapcsolatos követelmények}
%\comment{A szoftver fejlesztésével és használatával kapcsolatos számítógépes, hardveres, alapszoftveres és egyéb architekturális és logisztikai követelmények}

\begin{kovetelmeny}
	{E01} %Azonosító
	{Magas} %Prioritás
	{University} %Forrás
	{} %Ellenőrzés
	{A Szoftvernek a kijelölt virtuális gépen le kell fordulnia és azon futnia is kell.} %Leírás
    Ez a VM a kari felhőben (smallville) lesz elérhető.
\end{kovetelmeny}

\begin{kovetelmeny}
	{E02} %Azonosító
	{Magas} %Prioritás
	{University} %Forrás
	{} %Ellenőrzés
	{Maximum Java 11 használható fejlesztésre.} %Leírás
    Java 11 verzió lesz telepítve a kijelölt virtuális gépekre.
\end{kovetelmeny}

\begin{kovetelmeny}
	{E03} %Azonosító
	{Magas} %Prioritás
	{University} %Forrás
	{} %Ellenőrzés
	{A játék futtatásához egy darab számítógépnek ekégnek kell lennie.} %Leírás
    Minden esetben kell tudni teszteni a játékot akár internetkapcsolat nélkül is. Ez nem vonatkozik a fejleszésre és fordításra.
\end{kovetelmeny}

\subsection{Átadással kapcsolatos követelmények}
%\comment{A szoftver átadásával, telepítésével, üzembe helyezésével kapcsolatos követelmények}
\begin{kovetelmeny}
    {01} %Azonosító
    {Közepes} %Prioritás
    {Csapat} %Forrás
    {} %Ellenőrzés
    {A kész programnak minél kevesebb file-ból kell állnia.} %Leírás

\end{kovetelmeny}

\begin{kovetelmeny}
    {02} %Azonosító
    {Magas} %Prioritás
    {Forrás} %Forrás
    {Ellenőrzés} %Ellenőrzés
    {Leírás} %Leírás

\end{kovetelmeny}

\subsection{Egyéb nem funkcionális követelmények}
%\comment{A biztonsággal, hordozhatósággal, megbízhatósággal, tesztelhetőséggel, a felhasználóval kapcsolatos követelmények}
\begin{kovetelmeny}
    {Azonosító} %Azonosító
    {Pioritás} %Prioritás
    {Forrás} %Forrás
    {Ellenőrzés} %Ellenőrzés
    {Leírás} %Leírás

\end{kovetelmeny}

\section{Lényeges use-case-ek}
%\comment{Funkcionális követelmények részben felsorolt követelmények közül az alapvető és fontos követelményekhez tartozó használati esetek megadása az alábbi táblázatos formában.}
\subsection{Use-case leírások}
%\comment{Minden use-case-hez külön}

\begin{use-case}
	{Action}
	{Tartalmazza a játékos által irányított telepesek összes cselekvését.}
	{Player}
    \textbf{A.1} Minden tartalmazott tevékenység egy lépésnyi időt vesz igénybe. \newline
	\textbf{A.2} A játékos dönthet úgy, hogy nem csinál semmit.
\end{use-case}

\begin{use-case}
	{Move}
	{Tartalmazza a telepesek mindenféle mozgását.}
	{Player}
    \textbf{A.1} Egy telepes tud szomszédos aszteroidára utazni.
\end{use-case}

\begin{use-case}
	{Use gate}
	{A teleportkapu használata.}
	{Player}
    \textbf{A.1} Egy telepes tudja használni a kaput, a rajta lévő aszteroidán, egy lépésként beléphet a kapuba, majd kijön a másik oldalon.  
\end{use-case}

\begin{use-case}
	{Drill}
	{Az aszteroida fúrása.}
	{Player}
    \textbf{A.1} A telepes egy egységnyivel tud mélyebbre fúrni a köpenyben, amíg az aszteroida köpenye át nincs fúrva teljesen.
	\newline
	\textbf{A.2} Ha az aszteroida köpenye már teljesen át van fúrva, akkor nem lehet fúrni.
\end{use-case}

\begin{use-case}
	{Mine}
	{A telepesnek bányászása.}
	{Player}
    \textbf{A.1} A telepes akkor tud bányászni, ha az aszteroida köpenye már teljesen át van fúrva. Ekkor a telepes egy nyersanyagot kap és a mag tartalma eggyel csökken.
	\newline
	\textbf{A.2} A telepes nem tud bányászni, ha az aszteroida magja üreges.
	\newline
	\textbf{A.3} A telepes nem tud bányászni, ha van már nála 10 nyersanyag.
\end{use-case}

\begin{use-case}
	{Place gate}
	{Egy teleportkapu lehelyezése.}
	{Player}
    \textbf{A.1} Amennyiben kevesebb, mint 5 kapu van az aszteroidán és a telepesnél van teleportkapu, akkor lerakhat egyet.
	\newline
	\textbf{A.1.a} Ha a párnak az első kapuját helyezte le, akkor nem aktiválódik.
	\newline
	\textbf{A.1.b} Ha a párnak a második kapuját helyezte le, akkor létrejön a kapcsolat, és használhatóvá válnak a kapuk. 
\end{use-case}

\begin{use-case}
	{Place resource}
	{A kinyert nyersanyag visszarakása.}
	{Player}
    \textbf{A.1} Ha az aszteroida magja üres, akkor a telepes visszahelyezhet bele nyersanyagot.
	\newline
	\textbf{A.2} Minden aszteroidában csak annyi nyersanyag tárolható, amennyi hely van benne.
\end{use-case}

%TODO: talk it over
\begin{use-case}
	{Direct robot *}
	{A robotok irányítása.}
	{Player}
    \textbf{A.1} A telepes bármikor irányíthatja a saját készített robotokat, vagy használ vándorlási módot, vagy a kíséretes algoritmust. 
	\newline
	\textbf{A.2} A kíséretes algoritmus használat esetén kísérel a tulajdonost, és mindig abban vagy a szomszédos aszteroidán marad és fúr.
\end{use-case}

\begin{use-case}
	{Hide}
	{A telepes elbújása.}
	{Player}
    \textbf{A.1} A napvihar érkezéskor a telepes elbújhat egy üreges aszteroidába, akkor, ha teljesen át van fúrva a köpenye.
\end{use-case}

%TODO: meg kepp beszélni.
\begin{use-case}
	{Craft}
	{Mindenféle készítés.}
	{Player}
    \textbf{A.1} Ha a telepes rendelkezésére áll  éppen a telepes megtartott nyersanyag és az aszteroida magban tartott megszerzett nyersanyag együttes megfelel valamelyik képletet, akkor készíteni lehet készítményet. 
\end{use-case}

\begin{use-case}
	{Build robot}
	{Egy robot készítése.}
	{Player}
    \textbf{A.1} Egy egység vas, egy egység szén és egy egység urán felhasználásával egy lépésként készíthet a telepes egy autonóm robotot.
	\newline
	\textbf{A.2} A robot elkészülésekor automatikusan elhelyezi a telepes rajta lévő aszteroidán, és következő lépéstől kezdi vándorlási módon dolgozni.
\end{use-case}

\begin{use-case}
	{Build gate}
	{Egy teleportkapu készítése.}
	{Player}
    \textbf{A.1} Két egység vas, egy egység vízjég és egy egység urán felhasználásával egy lépésként lehet felépíteni egy teleportkapu-párt.
\end{use-case}

\begin{use-case}
	{Build station}
	{A bázis készítése.}
	{Player}
    \textbf{A.1} Ha a telepesek összegyűjtenek minden nyersanyagból legalább 3 egységet, és tárolják ezeket egy aszteroidában, akkor a nyersanyagok felhasználásával készíthetnek egy bázist.
	\newline
	\textbf{A.2} A bázist elkészüléskor automatikusan elhelyezi a telepes azon az aszteroidán, amin éppen tartózkodik.
\end{use-case}

\begin{use-case}
	{View map}
	{A térkép megtekintése.}
	{Player}
    \textbf{A.1} A játékos bármikor megnézheti a már meglátogatott aszteroidákat ábrázoló térképet. Ez nem kerül lépésbe.
\end{use-case}

\begin{use-case}
	{View inventory}
	{A tároló megtekintése.}
	{Player}
    \textbf{A.1} A játékos bármikor megnézheti a kiválasztott telepes raktárában lévő anyagokat (típusok és mennyiségek).
	\newline
	\textbf{A.2} A játékos bármikor megnézheti a kiválasztott aszteroidában lévő anyagokat (típusok és mennyiségek).
\end{use-case}

\begin{use-case}
	{Control robot}
	{A térkép megtekintése.}
	{Controller}
    \textbf{A.1} A mesterséges intelligencia a szerinte megfelelő lépéseket végezteti a robotokkal körönként.
	\newline
	Lehetséges cselekvések:
	\begin{itemize}
		\item Mozgás
		\item Fúrás
		\item Elbújás
	\end{itemize}
		
\end{use-case}

\begin{use-case}
	{Solar flare}
	{A napvihar vezérlése.}
	{Controller}
    \textbf{A.1} Vezérli a napvihart, változó időközönként történik egyszer, elér az összes aszteroidára, megölhet telepeseket és robotokat, akik nem bújtak el.
\end{use-case}

\begin{use-case}
	{Move asteroid field}
	{A aszteroidaöv mozgása.}
	{Controller}
    \textbf{A.1} A Controller mozgatja az aszteroidamezőt, azt a Naptól távolabbra vagy közelebbre mozgatja.
	\newline
	\textbf{A.1.a} Egy bizonyos Naptól való távolságnál az napközelinek számít.
\end{use-case}

\begin{use-case}
	{Check end condition}
	{A megnyerési feltétel ellenőrzése.}
	{Controller}
    \textbf{A.1} Ha legalább egy bázis le van helyezve, akkor a játékosok megnyerik a játékot.
	\newline
	\textbf{A.2} Ha minden telepes meghal, akkor a játékosok veszítenek, és a játék végét ér.
\end{use-case}






\subsection{Use-case diagram}
\begin{figure}[H]
	\includegraphics[width=1\textwidth]{\subfix{use-case.eps}}
	\centering
\end{figure}

\section{Szótár}
%\comment{A szótár a követelmények alapján készítendő fejezet. Egy szótári bejegyzés definiálásához csak más szótári bejegyzések és köznapi – a feladattól független – fogalmak használhatók fel. A szótár mérete kb. 1-2 oldal legyen. A bejegyzések legyenek ABC sorrendben!}

%szotar.sh script, lsd. a dokumentációt!!!
\begin{szotar}
	
    \szotaritem{Telepes }{A játékos által irányított ember, aki űrhajóval utazik és dolgozik az aszteroidákon.}
	\szotaritem{Aszteroida }{Az aszteroidaövben lévő kőhalmaz, van köpenye és magja. A telepesek és robotok mellete állomásoznak.}
    \szotaritem{Köpeny }{Az aszteroida külső rétege, van bizonyos mélysége. A telepesek és robotok tudnak belefúrni lyukat.}
	\szotaritem{Mag }{Az aszteroida belseje, lehet benne természetes nyersanyagot találni, de üres is lehet.}
	\szotaritem{Nyersanyag }{Az aszteroida magjában található természetes erőforrás, sok fajta létezik, lehet radioaktív is.}
	\szotaritem{Radioaktív }{A nyersanyag egy veszélyes tulajdonsága. Napközelben felrobbanhat.}

	\szotaritem{Vas }{Egy fajta nyersanyag.}
	\szotaritem{Vízjég }{Egy fajta nyersanyag. Napközelben elpárologhat.}
	\szotaritem{Szén }{Egy fajta nyersanyag.}
	\szotaritem{Urán }{Egy fajta nyersanyag, nagyon radioaktív.}

	\szotaritem{Napvihar  }{Csillagban létrejött természetes periódusos jelenség, amely amikor eléri az aszteroidamezőt, megölheti a telepeseket és robotokat.}

	\szotaritem{Robot  }{Nyersanyagból készült autonóm robot, segítséget ad a telepes munkája során. }

\end{szotar}


\section{Projekt terv}
\comment{Tartalmaznia kell a projekt végrehajtásának lépéseit, a lépések, eredmények határidejét, az egyes feladatok elvégzéséért felelős személyek nevét és beosztását, a szükséges erőforrásokat, stb. Meg kell adni a csoportmunkát támogató eszközöket, a választott technikákat! Definiálni kell, hogy hogyan történik a dokumentumok és a forráskód megosztása!}


\subsection{Projektütemterv}

\begin{center}
	\begin{terv_v2}
			\tervitem{febr. 4.}  {Követelmény, projekt, funkcionalitás}{ Sike Ádám }
			\tervitem{márc. 2. } { Analízis modell kidolgozása 1. - beadás }{ Dömötör Péter } 
			\tervitem{márc. 9. } { Analízis modell kidolgozása 2. - beadás }{ Dömötör Péter }
			\tervitem{márc. 16. }{ Skeleton tervezése - beadás }{ Tatai Titusz Miklós }
			\tervitem{márc. 23. }{ Skeleton - beadás és a forráskód herculesre való feltöltése }{ Gao Tong }
			\tervitem{márc. 30. }{ Prototípus koncepciója - beadás }{ Nagy Beáta }
			\tervitem{ápr. 6. }{ Részletes tervek - beadás }{ Gao Tong }
			\tervitem{ápr. 27. }{ Prototípus - beadás és a forráskód, a tesztbemenetek és az elvárt kimenetek herculesre való feltöltése }{ Nagy Beáta }
			\tervitem{ máj. 4. }{ Grafikus felület specifikációja - beadás }{ Sike Ádám }
			\tervitem{ máj. 18. }{ Grafikus változat és Összefoglalás - beadás és a forráskód herculesre való feltöltése }{ Dömötör Péter } 
	\end{terv_v2}
\end{center}

\subsection{Erőforrások, eszközök}
A fejlesztés során felhasznált segédeszközök:
\begin{itemize}
	\item Dokumentáció: dokumentáló eszközök felsorolása
	\item Kommunikáció: kommunikációs platformok...
	\item Modellező eszköz: modellezési eszközök
	\item Fejlesztő környezetek: ...
	\item Forráskód megosztás, verziókezelés: ...
	\item Egyéb ....
\end{itemize}

\comment{Még szabadon felvehető releváns idetartozó dolgok...}

\clearpage

\begin{naplo}
	\naplotag{2021.02.12. 17:00 }{ 1 óra }{ Csapat }
	{ Értekezlet
		\newline Döntés: Ezen a meetingen beszéltük meg az alapvető dolgokat, mint a csapatnév és a stratégia.
		Részletesen \ref{appendix:meeting0}
	}
	\naplotag{2021.02.17. 10:00 }{ 2 óra }{ Csapat }
	{ Értekezlet
		\newline Az elő beadás feladatainak kiosztása, és a feladat további pontosítása
		Részletesen \ref{appendix:meeting1}
	}


	\naplotag{2021.02.17. 14:00 }{ 5 óra }{ Gao Tong }
	{ 
		Funkciális követelmények (1.3.1)
	}
	\naplotag{2021.02.19. 19:00 }{ 4 óra }{ Gao Tong }
	{ 
		Funkciális követelmények , Use-case leírás (1.3.1, 1.4.1)
	}
	\naplotag{2021.02.20. 19:00 }{ 3 óra }{ Gao Tong }
	{ 
		Funkciális követelmények, Szótár (1.3.1, 1.5)
	}



	\naplotag{2021.02.20. 12:00 }{ 1.5 óra }{ Sike Ádám }
	{ 
		Funkciók leírása (1.2.2)
	}

	\naplotag{2021.02.20. 09:00 }{ 40 perc }{ Tatai Titusz Miklós }
	{ 
		Szakterület, hivatkozások, összefoglalás, projektterv előkészítése (1.1.2, 1.1.3, 1.1.5, 1.6)
	}

	\naplotag{2021.02.20. 11:00 }{ 1 óra }{ Nagy Beáta }
	{ 
		Áttekintés fejezet  (1.2.1, 1.2.3, 1.2.4, 1.2.5)
	}

	\naplotag{2021.02.20. 14:00 }{ 4 óra }{ Csapat }
	{ 
		Értekezlet
		\newline A mindenki által elkészített szöveg egyesítése és annak esetleges javítása.
		\ref{appendix:meeting2}
	}

\end{naplo}

\begin{toappendix}
	\markdownInput[shiftHeadings=2]{\subfix{docs/meetings/meeting_0.md}}

	\markdownInput[shiftHeadings=2]{\subfix{docs/meetings/meeting_1.md}}

	\markdownInput[shiftHeadings=2]{\subfix{docs/meetings/task_breakdown.md}}
\end{toappendix}

\end{document}
