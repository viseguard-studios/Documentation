\documentclass[../../projlab]{subfiles}
\begin{document}

\makeatletter

\ifSubfilesClassLoaded{
	\coverpage{7. Leadás}
	%\renewcommand{\filePath}[1]{./../../ #1}
	\def\filePath[#1]{./../../#1}
}{}

\makeatother

\chapter{Pontosítások}

\section{Teleport kapuk}
A teleport kapuk miután megkergülnek csak a fizikailag szomszédos aszteroidákra tud átmenni.
Másik teleport kaput nem tud használni illetve saját magát sem tud keresztül menni.
Ehhez az Aszteroidának kell egy új medódust adni, \verb"GetPhisycalNeighbours()" ami a fizikalilag szomszédos aszteroidákat adja vissza

\section{Aszteroidák}
Szükség van egy \verb"bool" változóra annak eltárolására, hogy ki volt-e bányászva. 
Így miután ki van báynászva lehet más dolgokat is eltárolni benne.



\section{Bemeneti nyelv}

\subsubsection{Bemeneti parancsok}

\begin{itemize}
    \item help
    \begin{itemize}
        \item Leírás: Kilistázza az összes lehetséges parancsot
        \item Opciók: help
    \end{itemize}


    \item startgame
    \begin{itemize}
        \item Leírás: 
        \item Opciók: startgame <seed> <file name>
        \begin{itemize}
            \item seed: a véletlen szám generátor által hasznát kezdőérték. \newline
Ha 0-t adnak meg a seednek akkor a játék semmilyen véletlenszerű tevékenységet nem fog végezni
            \item file name: A felhasználó adja meg a játéktér leírását tartalmazó file-t
        \end{itemize}
    \end{itemize}

    \item move
    \begin{itemize}
        \item Leírás: A játékos egy járművet akar mozgatni. \newline
            A parancs megkérdezi a játékostól, hogy melyik aszteroidára akarja mozgatni a járművet.
        \item Opciók: move <vessel>
        \begin{itemize}
            \item vessel: A jármű amit mozgatni akarunk
        \end{itemize}
        \item Példa:
            \begin{verbatim}
> move ss_001
Neighbours:
01 - Asteroid_1  - ?
02 - Asteroid_34 - Iron
03 - Asteroid_9  - Coal
04 - Asteroid_11 - ?
> 04
ss_001 Moved to Asteroid_11
            \end{verbatim}
    \end{itemize}


    \item neighbours
    \begin{itemize}
        \item Leírás: A játékos lekéri az aszteroida szomszédjait
        \item Opciók: neighbours <asteroid>
        \begin{itemize}
            \item asteroid: Az aszteroida aminek a szomszédjait akarjuk lekérdezni.
        \end{itemize}
        \item Példa:
            \begin{verbatim}
> neighbours ss_001
Neighbours:
01 - Asteroid_1  - ?
02 - Asteroid_34 - Iron
03 - Asteroid_9  - Coal
04 - Asteroid_11 - ?
            \end{verbatim}
    \end{itemize}


    \item drill
    \begin{itemize}
        \item Leírás: A játékos a megadott járművel fúr azon az aszteroidán amin éppen van.
        \item Opciók: drill <vessel>
        \begin{itemize}
            \item vessel: A jármű amivel fúrni akarunk.
        \end{itemize}
        \item Példa 1:
            \begin{verbatim}
> drill ss_001 
Asteroid_42 crust = X
            \end{verbatim}
        \item Példa 2:
            \begin{verbatim}
> drill ss_001 
can't drill
            \end{verbatim}
    \end{itemize}


    \item mine
    \begin{itemize}
        \item Leírás: A játékos a megadott járművel megpróbál bányászni azon az aszteroidán amin éppen van.
        \item Opciók: mine <vessel>
        \begin{itemize}
            \item vessel: A jármű amivel bányászni akarunk.
        \end{itemize}
        \item Példa:
            \begin{verbatim}
> mine ss_001
Asteroid_35 mined
* Item: Iron
            \end{verbatim}
    \end{itemize}


    \item status
    \begin{itemize}
        \item Leírás: A játékos a megadott entitás jelenlegi állapotát kérdezi le.
        \item Opciók: status <vessel|asteroid>
        \begin{itemize}
            \item vessel: A jármű aminek az állapotát kérdezzük le.
            \item asteroid: Az aszteroida aminek az állapotát kérdezzük le.
        \end{itemize}
        \item Példa 1:
            \begin{verbatim}
> status ss_001
Turn used: false
Asteroid: Asteroid_67 | Not hidden
Resources:
- Iron
- Iron
- Coal
- etc....
Teleport Gates:
- Tg_023 - 1
- Tg_023 - 1
- Tg_011 - 5
            \end{verbatim}
        \item Példa 2:
            \begin{verbatim}
> status Asteroid_67
Crust: 0
Vessels:
- SpaceShip_01
- SpaceShip_23
- Robot_4
Resources:
- Iron
- Iron
Teleport Gates:
- Tg_023 - 1
- Tg_011 - 5
            \end{verbatim}        
    \end{itemize}


    \item hide
    \begin{itemize}
        \item Leírás: A játékos a megadott járművet elbjutatja az aszteroidában anin tartózkodik.
        \item Opciók: hide <enter|exit> <vessel>
        \begin{itemize}
            \item vessel: A jármű aminek az állapotát kérdezzük le.
        \end{itemize}
        \item Példa:
            \begin{verbatim}
> hide enter ss_001
Can't hide
            \end{verbatim}
        \item Példa:
            \begin{verbatim}
> hide enter ss_001
ss_001 is hidden
            \end{verbatim}            
    \end{itemize}


    \item craft
    \begin{itemize}
        \item Leírás: A játékos egy űrhajón legyárt egy adott receptet.
        \item Opciók: craft <spaceShip> <recipe>
        \begin{itemize}
            \item spaceShip: Az űrhajó amin akarunk gyártani
            \item recipe: A recept amit le akarunk gyártani
        \end{itemize}
        \item Példa:
            \begin{verbatim}
> craft ss_001 <recipe>
Can't craft .......
--
Succesful
            \end{verbatim}
    \end{itemize}


    \item place
    \begin{itemize}
        \item Leírás: A játékos egy űrhajóból lerak egy nyersanyagot az aszteroidába amin van.
        \item Opciók: place <resource> from <vessel>
        \begin{itemize}
            \item resource: a nyersanyag amit le akarunk rakni
            \item vessel: A jármű amin akarunk gyártani
        \end{itemize}
        \item Példa:
            \begin{verbatim}
> palce Iron from ss_001
            \end{verbatim}
    \end{itemize}

    \item pickup
    \begin{itemize}
        \item Leírás: A játékos egy űrhajóból lerak egy nyersanyagot az aszteroidába amin van.
        \item Opciók: pickup <resource> to <vessel>
        \begin{itemize}
            \item resource: a nyersanyag amit le akarunk rakni
            \item vessel: A jármű amin akarunk gyártani
        \end{itemize}
        \item Példa:
            \begin{verbatim}
> pickup Coal to ss_001
            \end{verbatim}
    \end{itemize}


    \item endturn
    \begin{itemize}
        \item Leírás: A játékos befejezi a kört, ezzel a következő játékosnak adja át a kört.
        \item Opciók: endturn
        \item Példa:
            \begin{verbatim}
> endturn
Next player: player_2 
            \end{verbatim}
    \end{itemize}

    \item solarFlare
    \begin{itemize}
        \item Leírás: Beütemez egy napvihart a kör végére.
        \item Opciók: solarFlare <pos> <radius>
            \begin{itemize}
                \item pos: A napvihar pozíciója.
                \item radius: A napvihar ebben a körben hat.
            \end{itemize}
        \item Példa:
            \begin{verbatim}
> solarFlare 20 10
A solar flare will happen at 20,10
            \end{verbatim}
    \end{itemize}

    \item solarDistance
    \begin{itemize}
        \item Leírás: Módosítja a naptól való távolságot.
        \item Opciók: solarDistance <distance>
            \begin{itemize}
                \item distance: A napvihar pozíciója.
            \end{itemize}
        \item Példa:
            \begin{verbatim}
> solarDistance 10
            \end{verbatim}
    \end{itemize}

\end{itemize}

A robotok és UFOk irányítására az álltaluk végezhető parancsok használhatóak: \newline
Robotok
\begin{itemize}
    \item move
    \item drill
\end{itemize}
UFO
\begin{itemize}
    \item move
    \item mine
\end{itemize}

\chapter{Részletes tervek}
\comment{A dokumentum célja, hogy pontosan specifikálja az implementálandó osztályokat, beleértve a privát attribútumokat és metódusokat, ezek definícióját is.A dokumentum második fele részletesen be kell mutassa a korábban definiált be-és kimeneti nyelv szintakszisát felhasználva, hogy mely tesztekkel lesz a prototípus ellenőrizve.}

\section{Osztályok és metódusok tervei}

\subsection{AsteroidInventory}
\begin{class-template-responsibility}
 Egy aszteroida raktáráért (visszahelyezett kibányászott nyersanyagok tárolásáért) felelős osztály. 
\end{class-template-responsibility}
\begin{class-template-interface}
IInventory
\end{class-template-interface}
\begin{class-template-baseclass}
$\emptyset$
\end{class-template-baseclass}
\begin{class-template-attribute}
\classitem{-items:Item[0..*]}{A raktár elemeinek tárolója.}
\classitem{-capacity:int}{A raktár kapacitása.}
\end{class-template-attribute}
\begin{class-template-method}
\classitem{+insertItem(item:Item):bool}{Új elem hozzáadása a tárolóhoz. Az adott elemet hozzáadja az itemek listájához.}
\classitem{+removeItem(item:Item):bool}{Elem eltávolítása a tárolóból, amennyiben ez lehetséges. Az items listán addig iterál, amíg meg nem találja a keresett elemet (az item Satisfies metódusa igazzal tér vissza). Ha megtalálja, eltávolítja a listából, megszakítja az iterációt és igazzal tér vissza. Ha pedig az egész raktárban nem találta meg, hamissal tér vissza, nem sikerült eltávolítani semmit. }
\classitem{+tryInsertItem(item:Item):bool}{Ellenőrzi, hogy az adott elem elméletileg belehelyezhető-e a raktárba. Ellenőrzi, hogy a raktár kapacitása nagyobb-e, mint az aktuális mérete. Ha nem nagyobb, nem helyezhető bele az elem, hamissal tér vissza. Ha pedig igen, igazzal tér vissza, sikeres lenne a művelet. }
\classitem{+getItems():List<Item>}{Visszatér az inventory-ban található item-ek listájával.}
\classitem{NearSun()}{Napközelben hívódik meg, az items minden elemének NearSun metódusát meghívja (ez csak urán esetén fejt ki hatást).}    
\end{class-template-method}

\subsection{Building}
\begin{class-template-responsibility}
Az épülettípusok közös ősosztálya. 
\end{class-template-responsibility}
\begin{class-template-interface}
$\emptyset$
\end{class-template-interface}
\begin{class-template-baseclass}
Entity \baseclass MovableEntity
\end{class-template-baseclass}
\begin{class-template-attribute}
\item[] $\emptyset$
\end{class-template-attribute}
\begin{class-template-method}
\classitem{+getRoutes():Asteroid[0..*]}{Üres metódus, a teleportkapuk fogják felüldefiniálni a belőlük elérhető extra aszteroidákkal.}
\end{class-template-method}

\subsection{Coal}
\begin{class-template-responsibility}
 Egy egységnyi szén nyersanyagot reprezentál.
\end{class-template-responsibility}
\begin{class-template-interface}
$\emptyset$
\end{class-template-interface}
\begin{class-template-baseclass}
Resource \baseclass Item
\end{class-template-baseclass}
\begin{class-template-attribute}
\item[] $\emptyset$
\end{class-template-attribute}
\begin{class-template-method}
\classitem{+Satisfies(i:Item):bool}{Meghatározza, hogy az átadott Item használható-e a jelenlegi helyett. Beépített típusellenőrző függvénnyel megállapítja, hogy megegyeznek-e a típusok, és ha igen, true-val, ha nem, false-al tér vissza.}
\end{class-template-method}

\subsection{Ice}
\begin{class-template-responsibility}
 Egy egységnyi vízjég nyersanyagot reprezentál.
\end{class-template-responsibility}
\begin{class-template-interface}
$\emptyset$
\end{class-template-interface}
\begin{class-template-baseclass}
Resource \baseclass Item
\end{class-template-baseclass}
\begin{class-template-attribute}
\item[] $\emptyset$
\end{class-template-attribute}
\begin{class-template-method}
\classitem{+Satisfies(i:Item):bool}{Meghatározza, hogy az átadott Item használható-e a jelenlegi helyett. Beépített típusellenőrző függvénnyel megállapítja, hogy megegyeznek-e a típusok, és ha igen, true-val, ha nem, false-al tér vissza.}
\end{class-template-method}

\subsection{IInventory}
\begin{class-template-responsibility}
Interfész, ami a különböző raktártípusoktól elvárt megvalósítandó metódusokat gyűjti össze. 
\end{class-template-responsibility}
\begin{class-template-interface}
$\emptyset$
\end{class-template-interface}
\begin{class-template-baseclass}
$\emptyset$
\end{class-template-baseclass}
\begin{class-template-attribute}
\item[] $\emptyset$
\end{class-template-attribute}
\begin{class-template-method}
\classitem{+insertItem(item:Item):bool}{Új elem hozzáadása a tárolóhoz, amennyiben van benne szabad hely (kapacitás).  Sikeres művelet esetén igaz, sikertelen művelet esetén hamis visszatérési értéke van. }
\classitem{+removeItem(item:Item):bool}{Elem eltávolítása a tárolóból, amennyiben ez lehetséges. }
\classitem{+tryInsertItem(item:Item):bool}{Ellenőrzi, hogy az adott elem elméletileg belehelyezhető-e a raktárba. }
\classitem{+getItems():List<Item>}{Visszatér az inventory-ban található item-ek listájával.}
\classitem{NearSun()}{Napközelben hívódik meg, minden elem NearSun metódusát meghívja.}
\end{class-template-method}

\subsection{InfiniteInventory}
\begin{class-template-responsibility}
 Egy UFO végtelen tárhelyű raktáráért felelős osztály. 
\end{class-template-responsibility}
\begin{class-template-interface}
IInventory
\end{class-template-interface}
\begin{class-template-baseclass}
$\emptyset$
\end{class-template-baseclass}
\begin{class-template-attribute}
\classitem{-items:Item[0..*]}{A raktár elemeinek tárolója.}
\end{class-template-attribute}
\begin{class-template-method}
\classitem{+insertItem(item:Item):bool}{Új elem hozzáadása a tárolóhoz: az items listához hozzáadja a paraméterként megkapott elemet. Mindig igazzal tér vissza, hiszen nem kell méretet ellenőriznie, bármennyi elemet tud tárolni. }
\classitem{+removeItem(item:Item):bool}{Elem eltávolítása a tárolóból, amennyiben ez lehetséges. Az items listán addig iterál, amíg meg nem találja a keresett elemet (az item Satisfies metódusa igazzal tér vissza). Ha megtalálja, eltávolítja a listából, megszakítja az iterációt és igazzal tér vissza. Ha pedig az egész raktárban nem találta meg, hamissal tér vissza, nem sikerült eltávolítani semmit. }
\classitem{+tryInsertItem(item:Item):bool}{Ellenőrzi, hogy az adott elem elméletileg belehelyezhető-e a raktárba. Nincs művelet, igazzal tér vissza, hiszen végtelen hely van a raktárban, mindig sikeres lesz az elemek felvétele. }
\classitem{+getItems():List<Item>}{Visszatér az inventory-ban található item-ek listájával.}
\classitem{NearSun()}{Napközelben hívódik meg, az items minden elemének NearSun metódusát meghívja (ez csak urán esetén fejt ki hatást).}    
\end{class-template-method}

\subsection{Iron}
\begin{class-template-responsibility}
 Egy egységnyi vas nyersanyagot reprezentál.
\end{class-template-responsibility}
\begin{class-template-interface}
$\emptyset$
\end{class-template-interface}
\begin{class-template-baseclass}
Resource \baseclass Item
\end{class-template-baseclass}
\begin{class-template-attribute}
\item[] $\emptyset$
\end{class-template-attribute}
\begin{class-template-method}
\classitem{+Satisfies(i:Item):bool}{Meghatározza, hogy az átadott Item használható-e a jelenlegi helyett. Beépített típusellenőrző függvénnyel megállapítja, hogy megegyeznek-e a típusok, és ha igen, true-val, ha nem, false-al tér vissza.}
\end{class-template-method}

\subsection{Item}
\begin{class-template-responsibility}
Raktárban tárolható elemtípusok gyűjtőosztálya. 
\end{class-template-responsibility}
\begin{class-template-interface}
$\emptyset$
\end{class-template-interface}
\begin{class-template-baseclass}
$\emptyset$
\end{class-template-baseclass}
\begin{class-template-attribute}
\classitem{-inventory[1]: IInventory}{?}
\end{class-template-attribute}
\begin{class-template-method}
\classitem{+Activate()}{Üres metódus, a TeleportGateItem fogja felüldefiniálni a teleportkapuk lerakásakor végrehajtandó műveletekkel.}
\classitem{+NearSun(a:Asteroid)}{Napközelben a nyersanyag típusának megfelelő műveletet hajt végre. Alapértelmezetten üres metódus, különleges esetekben felüldefiniálható.}
\classitem{+Satisfies(r:Resource):bool}{Absztrakt metódus, a leszármazottakban meghatározza, hogy az átadott Item használható-e a jelenlegi helyett.}
\end{class-template-method}

\subsection{Player}
\begin{class-template-responsibility}
Egy adott játékos reprezentációja. 
\end{class-template-responsibility}
\begin{class-template-interface}
$\emptyset$
\end{class-template-interface}
\begin{class-template-baseclass}
$\emptyset$
\end{class-template-baseclass}
\begin{class-template-attribute}
\classitem{-name:string}{A játékos neve.}
\classitem{-searching\_for:Resource[1]}{Az itt szereplő nyersanyagfajta után kutatnak a játékos robotjai.}
\end{class-template-attribute}
\begin{class-template-method}
\item[] $\emptyset$
\end{class-template-method}

\subsection{Recipe}
\begin{class-template-responsibility}
A telepesek által elkészíthető receptekért felel. Egy recept adott számú kibányászott nyersanyag felhasználásával elkészített dolgok pontos hozzávalóit, és az elkészült eredmény típusát tárolja. 
\end{class-template-responsibility}
\begin{class-template-interface}
$\emptyset$
\end{class-template-interface}
\begin{class-template-baseclass}
$\emptyset$
\end{class-template-baseclass}
\begin{class-template-attribute}
\classitem{-input:Item[1..*]}{Az aktuális recepthez szükséges hozzávalók tárolása. }
\end{class-template-attribute}
\begin{class-template-method}
\classitem{+canCraft(s:SpaceShip):bool}{Meghatározza, hogy egy adott raktárban és aszteroidán lévő kibányászott nyersanyagkészlet elegendő-e az adott recept elkészítéséhez. Először lekéri a készítő telepes és az aktuális aszteroida raktárának elemkészletét, és lemásolja azokat egy-egy listába, hogy az eredeti tartalom ne változzon. Majd a recept input-jának minden elemét összehasonlítja a két lista elemeivel, először a telepesében keresve, és ha nem talál, az aszteroidáéban keresi tovább. Amikor megtalál egy elemet, eltávolítja a listából. Ha minden szükséges elemet megtalált a két listában összesen, elkészíthető a recept. Ha legalább egy hozzávalóhoz nem talál raktárkészletet egyik listában sem,azonnal befejezi a műveleteket, false-al tér vissza, nem készíthető el a recept.}
\classitem{+craft(s:SpaceShip)}{Akkor hívódik meg ha ténylegesen le akarjuk gyártani ezt a receptet. Lekérdezi a telepes raktárát, illetve az aktuális aszteroidát, és annak raktárát. A recept minden elemét először megpróbálja eltávolítani a telepes raktárából (RemoveItem metódussal), ha sikertelen, megpróbálja ezt az elemet az aszteroida raktárából hasonlóképpen eltávolítani (valamelyiknek sikerülnie kell a kettő közül, különben nem hívódna meg a függvény). Az összes hozzávaló eltávolítása után a recept MakeResult metódusa hívódik meg, ami elkészíti magát a receptből előállított dolgot.}
\classitem{\#makeResult(s:SpaceShip)}{Létrehozza a kívánt terméket a receptből a leszármazottak esetében, itt absztrakt metódus.}
\end{class-template-method}

\subsection{Resource}
\begin{class-template-responsibility}
Egy adott aszteroidában tárolt egy egységnyi, bányászással kinyerhető nyersanyagok alaposztálya. 
\end{class-template-responsibility}
\begin{class-template-interface}
$\emptyset$
\end{class-template-interface}
\begin{class-template-baseclass}
Item
\end{class-template-baseclass}
\begin{class-template-attribute}
\item[] $\emptyset$
\end{class-template-attribute}
\begin{class-template-method}
\item[] $\emptyset$
\end{class-template-method}

\subsection{RobotRecipe}
\begin{class-template-responsibility}
Egy robot elkészítéséhez szükséges speciális recept típus. 
\end{class-template-responsibility}
\begin{class-template-interface}
$\emptyset$
\end{class-template-interface}
\begin{class-template-baseclass}
Recipe
\end{class-template-baseclass}
\begin{class-template-attribute}
\item[] $\emptyset$
\end{class-template-attribute}
\begin{class-template-method}
\classitem{\#makeResult(s:SpaceShip)}{Lekérdezi az aktuális telepeshez tartozó játékost, majd ezt, illetve az aktuális aszteroidát megadva a konstruktorban egy új robotot hoz létre. A robotot az aktuális aszteroidán helyezi el, az adott játékosé lesz.}
\end{class-template-method}

\subsection{SpaceStation}
\begin{class-template-responsibility}
A játékosok által megépítendő űrállomás épülettípust jelöli. Speciális tulajdonsága, hogy amikor megépül akkor a játék befejeződik.
\end{class-template-responsibility}
\begin{class-template-interface}
$\emptyset$
\end{class-template-interface}
\begin{class-template-baseclass}
Entity \baseclass MovableEntity \baseclass Building 
\end{class-template-baseclass}
\begin{class-template-attribute}
\item[] $\emptyset$
\end{class-template-attribute}
\begin{class-template-method}
\classitem{+SpaceStation(a:Asteroid)}{Az osztály konstruktora. Egy űrállomás felépítése a játék végével jár, ezért a konstruktor először hozzáadja önmagát, mint épületet az aktuális aszteroidához(addBuilding). Ezután a játéktet (scene) lekérdezi az aszteroidától, a játéktértől pedig a manager-t kérdezi le. A managernek az EndGame() metódusát meghívva jelzi, hogy a játék véget ért. } 
\classitem{+explode()}{Üres metódus, miután felépült egy űrállomás, a játéknak vége, ezért ez a metódus sosem hívódik meg, nincs következménye.}
\classitem{+solarFlare()}{Üres metódus, miután felépült egy űrállomás, a játéknak vége, ezért ez a metódus sosem hívódik meg, nincs következménye.}
\end{class-template-method}

\subsection{SpaceStationRecipe}
\begin{class-template-responsibility}
Űrállomás elkészítéséhez szükséges speciális recept típus. 
\end{class-template-responsibility}
\begin{class-template-interface}
$\emptyset$
\end{class-template-interface}
\begin{class-template-baseclass}
Recipe
\end{class-template-baseclass}
\begin{class-template-attribute}
\item[] $\emptyset$
\end{class-template-attribute}
\begin{class-template-method}
\classitem{\#makeResult(s:SpaceShip)}{Létrehozza az űrállomást: meghívja az űrállomás konstruktorát. A konstruktor pedig befejezi a játékot (lásd:SpaceStation osztály, konstruktor leírás).}
\end{class-template-method}

\subsection{SSInventory}
\begin{class-template-responsibility}
 Egy egy telepes raktáráért felelős osztály. 
\end{class-template-responsibility}
\begin{class-template-interface}
IInventory
\end{class-template-interface}
\begin{class-template-baseclass}
$\emptyset$
\end{class-template-baseclass}
\begin{class-template-attribute}
\classitem{-items:Item[0..*]}{A raktárban található kibányászott nyersanyagok tárolója.}
\classitem{-tgs:TeleportGateItem[0..3]}{A raktárban lévő teleportkapuk tárolója.}
\classitem{-capacity:int}{A raktár nem teleportkapu típusú item-einek kapacitása.}
\classitem{-tgCapacity:int}{A raktár teleportkapuinak kapacitása.}
\end{class-template-attribute}
\begin{class-template-method}
\classitem{+insertItem(item:Item):bool}{Új elem hozzáadása a tárolóhoz. Az adott elemet hozzáadja az itemek listájához.}
\classitem{+removeItem(item:Item):bool}{Elem eltávolítása a tárolóból (itemek listájából), amennyiben ez lehetséges. A listán addig iterál, amíg meg nem találja a keresett elemet (az item Satisfies metódusa igazzal tér vissza). Ha megtalálja, eltávolítja a listából, megszakítja az iterációt és igazzal tér vissza. Ha pedig az egész listában nem találta meg, hamissal tér vissza, nem sikerült eltávolítani semmit. }
\classitem{+insertGate(tg:TeleportGateItem):bool}{Új teleportkapu hozzáadása a tgs listához, amennyiben van hely. Ha van hely, igazzal, ha nincs, hamissal tér vissza.}
\classitem{+removeGate(tg:TeleportGateItem):bool}{Teleportkapu eltávolítása a tgs listából, amennyiben megtalálja (a listán végigiterálva, Satisfies metódussal). Amint megtalálja, kilép a ciklusból, eltávolítja és igazzal tér vissza. Ha az egész listában nem találta, sikertelen a művelet, hamissal tér vissza.}
\classitem{+tryInsertItem(item:Item):bool}{Ellenőrzi, hogy az adott elem elméletileg belehelyezhető-e a raktárba. Ellenőrzi, hogy a raktár kapacitása nagyobb-e, mint az aktuális mérete. Ha nem nagyobb, nem helyezhető bele az elem, hamissal tér vissza. Más esetben igazzal tér vissza, van hely elhelyezni. }
\classitem{+tryInsertGate(tg:TeleportGateItem):bool}{Ellenőrzi, hogy az adott teleportkapu elméletileg belehelyezhető-e a raktárba. Ellenőrzi, hogy a raktár kapacitása nagyobb-e, mint az aktuális mérete. Ha nem nagyobb, nem helyezhető bele a kapu, hamissal tér vissza. Más esetben igazzal tér vissza, van hely elhelyezni. }
\classitem{+getItems():List<Item>}{Visszatér az inventory-ban található item-ek listájával.}
\classitem{+NearSun()}{Napközelben hívódik meg, az items minden elemének NearSun metódusát meghívja (ez csak urán esetén fejt ki hatást).}    
\end{class-template-method}

\subsection{TeleportGate}
\begin{class-template-responsibility}
A teleportkapukat reprezentáló osztály.
\end{class-template-responsibility}
\begin{class-template-interface}
$\emptyset$
\end{class-template-interface}
\begin{class-template-baseclass}
Entity \baseclass MovableEntity \baseclass Building 
\end{class-template-baseclass}
\begin{class-template-attribute}
\classitem{-idList:Map<TeleportGate,int>}{Egy statikus lista, ami tartalmazza az eddig lerakott teleportkapukat azonosítókkal párosítva.}
\classitem{-gateId:int}{A teleportkapu azonosítója.}
\classitem{-pair:TeleportGate}{A teleportkapu párját tárolja.}
\end{class-template-attribute}
\begin{class-template-method}
\classitem{+getIdPair(id:int):TeleportGate}{Megkeresi az idList-ben az adott id-hez tartozó teleportkaput, ha van ilyen, és visszatér vele. Ha nincs találat, null-al tér vissza.}
\classitem{+getRoutes():Asteroid}{A teleportkapu lekérdezi a párjától, hogy melyik aszteroida tartozik hozzá. A párja visszaadja ezt az aszteroidát, így a metódus visszaadja, hogy melyik extra aszteroida érhető el a teleportkapun keresztül egy lépésben.}
\classitem{+PairDestroyed()}{Ha az aszteroida párja megsemmisül, meghívódik ez a metódus, jelezve, hogy a teleportkapu már nem használható. A teleportkapu párja (pair attribútum értéke) törlődik, és mivel már le van rakva a teleportkapu, nem állítható be új pár számára.}
\classitem{+explode()}{A teleportkapu felrobban: értesíti a szomszédját a PairDestroyed() metóduson keresztül, hogy megsemmisült, majd ő maga is használhatatlanná válik. A pair attribútum értéke törlődik, nem visszaállítható.}
\classitem{+solarFlare()}{Napszél éri a kaput, aminek hatására véletlen időközönként egy véletlenszerű, szomszédos aszteroidára mozog át. Egy véletlenszám-generátor létrehozása, és az időköz beállítása után végignézi a szomszédai listáját (lekéri az aszteroidától a listát, amiben benne van a teleportkapu párjához tartozó aszteroida is). Ezután ezek közül egyet egy véletlenszerű választási algoritmussal kiválaszt, eltávolítja önmagát az aktuális aszteroidáról (removeBuilding() metódussal), majd hozzáadja magát az új aszteroida épületeihez (addBuilding() metódussal). }
\classitem{+setPair(t:TeleportGate)}{A pair attribútum setter-e.}
\classitem{+addIdListItem(t:TeleportGate,id:int)}{Az idList-hez hozzáad egy új elemet, beállítva a TeleportGate és id tulajdonságait.}
\end{class-template-method}

\subsection{TeleportGateItem}
\begin{class-template-responsibility}
A még le nem rakott teleportkapukért felelős osztály, amiket a telepes a raktárában tárol. 
\end{class-template-responsibility}
\begin{class-template-interface}
$\emptyset$
\end{class-template-interface}
\begin{class-template-baseclass}
Item
\end{class-template-baseclass}
\begin{class-template-attribute}
\classitem{-id:int}{A teleportkapu-párok közös azonosítója.}
\end{class-template-attribute}
\begin{class-template-method}
\classitem{+Satisfies(i:Item):bool}{Meghatározza, hogy az átadott Item használható-e a jelenlegi helyett. Beépített típusellenőrző függvénnyel megállapítja, hogy megegyeznek-e a típusok, és ha igen, true-val, ha nem, false-al tér vissza.}
\classitem{+Activate()}{Teleportkapuk lerakásáért felel. Eltávolítja magát a raktárból. Ezután a teleportkapu osztály statikus idList listájából megkeresi, hogy a hozzá tartozó id-hez tartozik-e már teleportkapu (getIdPair metódus), és ha igen, megtalálta a párját, ami már el van helyezve. Ekkor létrehoz egy új teleportkaput, aminek a párját beállítja az id alapján megtalált kapura, aminél pedig beállítja saját magát párnak (setPair metódussal). Ha pedig nem találja a listában az id-t (null-al tér vissza a metódus),akkor létrehoz egy új teleportkaput, pár beállítás nélkül. Mindkét opció esetén végül elhelyezi magát az idListben (addIdListItem metódussal)}
\end{class-template-method}

\subsection{TeleportGateRecipe}
\begin{class-template-responsibility}
Teleportkapu-párok elkészítéséhez szükséges speciális recept típus. 
\end{class-template-responsibility}
\begin{class-template-interface}
$\emptyset$
\end{class-template-interface}
\begin{class-template-baseclass}
Recipe
\end{class-template-baseclass}
\begin{class-template-attribute}
\item[] $\emptyset$
\end{class-template-attribute}
\begin{class-template-method}
\classitem{\#makeResult(s:SpaceShip)}{Generál egy azonosítót az új teleportkapu-párnak, majd ezt megadva a konstruktorban 2 teleportkapu elemet hoz létre, majd elhelyezi őket a telepes raktárában (insertItem).}
\classitem{+canCraft(s:SpaceShip):bool}{Meghatározza, hogy egy adott raktárban és aszteroidán lévő kibányászott nyersanyagkészlet elegendő-e az teleportkapu-pár elkészítéséhez. Először lekéri a készítő telepes és az aktuális aszteroida raktárának elemkészletét, és lemásolja azokat egy-egy listába, hogy az eredeti tartalom ne változzon. Majd a recept input-jának minden elemét összehasonlítja a két lista elemeivel, először a telepesében keresve, és ha nem talál, az aszteroidáéban keresi tovább. Amikor megtalál egy elemet, eltávolítja a listából.  Ha legalább egy hozzávalóhoz nem talál raktárkészletet egyik listában sem,azonnal befejezi a műveleteket, false-al tér vissza, nem készíthető el a recept. Ha minden szükséges elemet megtalált a két listában összesen, ellenőrzi, van-e elég hely a teleportkapuk tárolására a telepes raktárában. Ehhez le kell kérdeznie a raktár teleportkapukat tároló listájának méretét. Ha van elegendő hely, a recept elkészíthető, ellenkező esetben nem (true, ill. false visszatérés).}
\end{class-template-method}

\subsection{Titanium}
\begin{class-template-responsibility}
 Egy egységnyi titán nyersanyagot reprezentál.
\end{class-template-responsibility}
\begin{class-template-interface}
$\emptyset$
\end{class-template-interface}
\begin{class-template-baseclass}
Resource \baseclass Item
\end{class-template-baseclass}
\begin{class-template-attribute}
\item[] $\emptyset$
\end{class-template-attribute}
\begin{class-template-method}
\classitem{+Satisfies(i:Item):bool}{Meghatározza, hogy az átadott Item használható-e a jelenlegi helyett. Beépített típusellenőrző függvénnyel megállapítja, hogy megegyeznek-e a típusok, és ha igen, true-val, ha nem, false-al tér vissza.}
\end{class-template-method}

\subsection{Uranium}
\begin{class-template-responsibility}
 Egy egységnyi urán nyersanyagot reprezentál.
\end{class-template-responsibility}
\begin{class-template-interface}
$\emptyset$
\end{class-template-interface}
\begin{class-template-baseclass}
Resource \baseclass Item
\end{class-template-baseclass}
\begin{class-template-attribute}
\classitem{-counter: int}{Számolja, hogy urán hányszor volt eddig napközelben, ettől függően robban fel.}
\end{class-template-attribute}
\begin{class-template-method}
\classitem{+Satisfies(i:Item):bool}{Meghatározza, hogy az átadott Item használható-e a jelenlegi helyett. Beépített típusellenőrző függvénnyel megállapítja, hogy megegyeznek-e a típusok, és ha igen, true-val, ha nem, false-al tér vissza.}
\classitem{+NearSun(a:Asteroid)}{Napközelben hívódik meg, a számlálót növeli 1-el. Ha az eléri a 3-at, felrobban az urán. Ez azt jelenti, hogy a megadott aszteroida explode() metódusa meghívódik, felrobbantva vele az aszteroidát.}
\end{class-template-method}


\section{A tesztek részletes tervei, leírásuk a teszt nyelvén}
\comment{A tesztek részletes tervei alatt meg kell adni azokat a bemeneti adatsorozatokat, amelyekkel a program működése ellenőrizhető. Minden bemenő adatsorozathoz definiálni kell, hogy az adatsorozat végrehajtásától a program mely részeinek, funkcióinak ellenőrzését várjuk és konkrétan milyen eredményekre számítunk, ezek az eredmények hogyan vethetők össze a bemenetekkel.A tesztek leírásakor az előző dokumentumban (proto koncepciója) megadott szintakszist kell használni.}

\subsection{Teszteset1}
\begin{test-case-description}
    Szöveges leírás, kb 1-5 mondat
\end{test-case-description}
\begin{test-case-function}
    ...
\end{test-case-function}
\begin{test-case-input}
    A proto nyelvén megadva, pl.:
    \begin{verbatim}
    init world
    0: S 3
    ...
    print
    \end{verbatim}
\end{test-case-input}
\begin{test-case-output}
    A poro kimeneti nyelvén megadva, pl.:
    \begin{verbatim}
    worlddata
    ...
    creatures 0
    \end{verbatim}
\end{test-case-output}

\section{A tesztelést támogató programok tervei}
%\comment{A tesztadatok előállítására, a tesztek eredményeinek kiértékelésére szolgáló segédprogramok részletes terveit kell elkészíteni.}
A program teszteléséhez a PowerShell parancssori értelmezőt javasoljuk használni.
Ehhez a PS stream átirányítás funkcióját illetve a különbség kijelzőt lehet felhasználni.
Egy példa a PS használatára:

\begin{adjustwidth}{-30pt}{0pt}
\begin{verbatim}
>Get-Content input.txt | java -jar asteroidMiner.jar > output.txt
>Compare-Object $(Get-Content "sample.txt") $(Get-Content "output.txt") -IncludeEqual
\end{verbatim}
\end{adjustwidth}

A példában a program egy előre elkészített bemenetet olvas be ( \verb"input.txt" )

\end{document}