\chapter{Összefoglalás}
\section{A projektre fordított összes munkaidő}
\comment{Személyenként és összesen}
\begin{ertekelesOra}
    \ertekelestag{Dömötör Péter}{ G2Y5TI }{110 óra, ebből 53 óra önálló és 57 óra csapatmunka}
    \ertekelestag{Sike Ádám}{ E8Z277 }{87 óra, ebből 30 óra önálló és 57 óra csapatmunka }
    \ertekelestag{Gao Tong}{ I2SVOS }{96 óra, ebből 39 óra önálló és 57 óra csapatmunka}
    \ertekelestag{Nagy Beáta}{ GPOGC5 }{98 óra, ebből 41 óra önálló és 57 óra csapatmunka}
    \ertekelestag{Tatai Titusz Miklós}{ IJHLYX }{77 óra, ebből 20 óra önálló és 57 óra csapatmunka}
    \ertekelestag{Csapatgyűlés}{ - }{ 57 óra}
    \ertekelestag{Összesen}{ - }{ $\sum$ }
\end{ertekelesOra}

\section{A feltöltött programok forrássorainak megoszlása}
\begin{ertekelesKod}
    \ertekelestagk{Szkeleton}{3179}
    \ertekelestagk{Prototípus}{5165}
    \ertekelestagk{Grafikus változat}{??}
    \ertekelestagk{Összesen}{ $\sum$  }
\end{ertekelesKod}

\section{Projekt összegzés}
\comment{A projekt tapasztalatait összegző részben a csapatoknak a projektről kialakult véleményét várjuk. A megválaszolandók köre az alábbi}
\begin{itemize}
\item \textit{Mit tanultak a projektből konkrétan és általában?} \newline
    -Fontos, hogy a projekt követelményeit, az elvárásokat pontosan megértsük, mert a félreértésekből rengeteg probléma adódhat később. 
    -A projektmunka megfelelő felépítése és interpretálása is fontos része a folyamatnak. A fejlesztés során rengeteg múlik a kód megfelelő kialakításán. Sokszor számunkra jónak tűnő, működő megoldások során elvi hibákat kaptunk visszajelzésként, így a megfelelő kialakításra a jövőben jobban oda kell figyelnünk. 
    -A csoportmunka hatékony megszervezése nagyon fontos lépés, nagyban meghatározza a projekt sikerességét, és az összes időbefektetést. 
    -A csoportnormák meghatározása, a csoporton belüli megegyezés is fontos tényező, figyelnünk kell rá, hogy olyan kompromisszumokat találjunk (lehetőleg minél hatékonyabban), ami mindenkinek megfelel.  
\item \textit{Mi volt a legnehezebb és a legkönnyebb?} \newline
    -Számunkra a legnehezebb feladatrész a prototípus leadása volt, ezzel töltöttük a legtöbb időt és itt volt szükség sokszor hibajavításra, kódrészletek vagy logika átalakítására. A legkönnyebb beadások pedig az első néhány hét során megalkotott tervek voltak, itt sok ötletünk volt és jó pontszámokat kaptunk. 
    -Általánosságban a határidők betartása, és az időmenedzsment ment nekünk a legnehezebben. A legkönnyebb pedig az ötletelés, a problémákra sokféle megoldás találása volt számunkra.
\item \textit{Összhangban állt-e az idő és a pontszám az elvégzendő feladatokkal?} \newline
    A pontszám összhangban volt a feladatok nehézségével, viszont az idő számunkra nem volt összhangban a feladatok komplexitásával, hosszával. Például volt olyan hét, amikor 10 pontos beadást kellett elkészítenünk, ahol lényegesen kevesebb feladat volt, mint egy 30-40 pontos beadás esetében. Az utóbbi esetben sokkal nehezebb volt jó pontszámot elérnünk, sokkal nagyobb időbefektetés kellett ugyanannyi idő alatt. A heti leadások közti arányon egy kicsit lehetne finomítani, összehangolni, hogy minden héten hasonló nehézségű, hasonló időt igénylő beadást készítsünk. 
\item \textit{Ha nem, akkor hol okozott ez nehézséget?} \newline
    A tárgyon kívül sok másik tárgyra kellett koncentrálnunk hétről hétre, így nem mindig jutott rá elég időnk. A tárgy 3 kreditet ér, és sokszor úgy éreztük, hogy a projekt ennél nagyobb komplexitású, és több időbefektetést igényel. Következő években 3 kreditre egy kicsit kevésbé komplex projektet, vagy több kreditért egy hasonló nagyságrendű projekt kidolgozását ajánljuk. 
\item \textit{Milyen változtatási javaslatuk van?} \newline
    -Sokat segített volna nekünk a leadások során, ha lett volna egy példaprojekt, amihez tartozik minden leadáshoz egy minta dokumentáció. Ezen keresztül pontosabban láttuk volna, hogy a dokumentáció egyes pontjainál helyesen értelmezzük, hogy pontosan miről kellene írnunk. 
    -A kódunk implementációja során sok olyan problémába ütköztünk, ami a tervezés során nem jött elő, nem tudtuk, hogy a gyakorlatban ez nem fog működni. Az UML modellek alkotása során azt tanultuk meg, hogy a tervezésnek minél inkább nyelvfüggetlennek kell lennie, hogy bármilyen környezetben implementálni tudjuk. A gyakorlatban viszont találkoztunk néhány olyan problémával, ami az adott nyelvi környezetben jön elő, és emiatt viszonylag nehéz volt módosítani a modell megfelelő részein. Ezért azt javasoljuk, hogy kicsit nagyobb hangsúly legyen az implementáción a részletes tervezés mellett. 
    -Egy-egy feladatrész (skeleton, prototípus, grafikus verzió) tesztelését manuálisan végeztük, ami során sok hiba nehezen, vagy egyáltalán nem jön elő. Ehelyett a javaslatunk valamilyen tesztosztály, pl. JUnit tesztek elvégzése lenne, ami egy átfogóbb, pontosabb képet adna a kód helyességéről, és jobban kiszűrné a potenciális hibalehetőségeket. 
\item \textit{Milyen feladatot ajánlanának a projektre?} \newline
    -Egy grafikusan kirajzolt sakkjáték (pl. 5D chess w/ multiverse time travel)
    -Egy négyzethálón játszódó kolóniaszimuláció: a pályán emberek mozognak, akikhez feladatokat rendelhetünk (pl. favágás, bányászás), amihez a pályán elhelyezünk alapanyagokat, ezeket kell összeszedniük a tevékenység elvégzéséhez. Bizonyos tevékenységek elvégzése után házakat tudnak építeni maguknak (Rimworld játékhoz hasonló konstrukció)
    -További játékötleteink: Civilization IV, C&C (régebbiek), Warcraft, Factorio játékok  
\item \textit{Egyéb kritika és javaslat} \newline
    -A tárgyfelelős, Goldschmidt Tanár Úr is tudna esetleg segíteni a projekt sikeres elkészítése érdekében: adhatna ötleteket a megvalósításhoz, írhatna esetleg gyakoribb hibákról és elkerülésükről, arról, hogy mire figyeljünk oda egy-egy beadás során stb. 
    -A projekt során a verziókezelés egy fontos tényező, amit sajnos az eddigi tárgyakban nem tanultunk. Ezért a tárgy első hetén hasznos lenne egy összefoglaló előadás a git használatáról. Elsősorban nem command line megvalósításra gondoltunk, hanem kezdők számára egyszerűbb és átláthatóbb GUI-s megvalósításokra. Ajánljuk a GitKraken, és a SourceTree programokat a verziókezeléshez.
\end{itemize}


