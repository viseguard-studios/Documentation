\documentclass[../../projlab]{subfiles}
\begin{document}

\makeatletter

\ifSubfilesClassLoaded{
	\coverpage{5. Leadás}
	\def\filePath[#1]{./../../#1}
}{}

\makeatother

\chapter{Szkeleton beadás}

\section{Fordítási és futtatási útmutató}
%\comment{A feltöltött program fordításával és futtatásával kapcsolatos útmutatás. Ennek tartalmaznia kell leltárszerűen az egyes fájlok pontos nevét, méretét byte-ban, keletkezési idejét, valamint azt, hogy a fájlban mi került megvalósításra.}

\subsection{Fájllista}

\begin{fajllista}
	\fajl{Asteroid.java}{10214 byte} {2021. 03. 22. 13:50:20} {}
\fajl{Engine.java}{615 byte} {2021. 03. 20. 13:48:06} {}
\fajl{Entity.java}{916 byte} {2021. 03. 22. 13:48:36} {}
\fajl{GameManager.java}{3703 byte} {2021. 03. 22. 12:20:42} {}
\fajl{Inventory.java}{3992 byte} {2021. 03. 22. 13:48:36} {}
\fajl{Main.java}{159 byte} {2021. 03. 20. 13:48:07} {}
\fajl{Player.java}{1069 byte} {2021. 03. 22. 12:20:42} {}
\fajl{Robot.java}{1232 byte} {2021. 03. 22. 13:45:56} {}
\fajl{Scene.java}{1848 byte} {2021. 03. 22. 13:48:36} {}
\fajl{SpaceShip.java}{3897 byte} {2021. 03. 22. 13:17:36} {}
\fajl{Vessel.java}{5116 byte} {2021. 03. 22. 13:45:55} {}
\fajl{Logger.java}{1916 byte} {2021. 03. 22. 10:22:18} {}
\fajl{Registry.java}{435 byte} {2021. 03. 20. 14:11:05} {}
\fajl{SkeletonEntry.java}{1860 byte} {2021. 03. 22. 13:48:36} {}
\fajl{STest.java}{212 byte} {2021. 03. 20. 13:35:59} {}
\fajl{Test.java}{203 byte} {2021. 03. 20. 13:48:07} {}
\fajl{Building.java}{1121 byte} {2021. 03. 20. 13:48:06} {}
\fajl{BuildingPlace.java}{124 byte} {2021. 03. 20. 13:48:06} {}
\fajl{SpaceStation.java}{2037 byte} {2021. 03. 22. 10:11:15} {}
\fajl{TeleportGate.java}{1119 byte} {2021. 03. 20. 13:48:06} {}
\fajl{Coal.java}{1003 byte} {2021. 03. 22. 11:59:05} {}
\fajl{Ice.java}{1002 byte} {2021. 03. 22. 11:59:05} {}
\fajl{Iron.java}{1005 byte} {2021. 03. 22. 11:59:05} {}
\fajl{Item.java}{1583 byte} {2021. 03. 22. 11:59:05} {}
\fajl{TeleportGateItem.java}{1314 byte} {2021. 03. 22. 11:59:05} {}
\fajl{Titan.java}{1011 byte} {2021. 03. 22. 11:59:05} {}
\fajl{Uranium.java}{1023 byte} {2021. 03. 22. 11:59:05} {}
\fajl{Recipe.java}{5733 byte} {2021. 03. 22. 13:21:00} {}
\fajl{RobotRecipe.java}{1274 byte} {2021. 03. 22. 13:19:29} {}
\fajl{SpaceStationRecipe.java}{1303 byte} {2021. 03. 22. 13:19:29} {}
\fajl{TeleportGateRecipe.java}{5566 byte} {2021. 03. 22. 13:21:00} {}
\fajl{CoalResource.java}{785 byte} {2021. 03. 22. 10:19:53} {}
\fajl{IceResource.java}{780 byte} {2021. 03. 22. 10:19:53} {}
\fajl{IronResource.java}{782 byte} {2021. 03. 22. 10:19:53} {}
\fajl{Resource.java}{1493 byte} {2021. 03. 22. 10:08:14} {}
\fajl{TitaniumResource.java}{795 byte} {2021. 03. 22. 10:19:53} {}
\fajl{UraniumResource.java}{798 byte} {2021. 03. 22. 10:19:53} {}
\fajl{CraftRobotTest.java}{2875 byte} {2021. 03. 22. 13:19:29} {}
\fajl{CraftSpaceStationTest.java}{3218 byte} {2021. 03. 22. 13:48:36} {}
\fajl{CraftTeleportGateTest.java}{3139 byte} {2021. 03. 22. 13:19:29} {}
\fajl{MoveAsteroidField.java}{980 byte} {2021. 03. 22. 12:20:42} {}
\fajl{QueueSolarFlareTest.java}{828 byte} {2021. 03. 22. 12:20:42} {}
\fajl{SolarFlareHitsRobot.java}{965 byte} {2021. 03. 22. 13:17:36} {}
\fajl{SolarFlareHitsSpaceShip.java}{985 byte} {2021. 03. 22. 13:17:36} {}
\fajl{SSDrillTest.java}{1119 byte} {2021. 03. 22. 10:08:14} {}
\fajl{SSExitHidingTest.java}{948 byte} {2021. 03. 22. 10:08:14} {}
\fajl{SSHideTest.java}{1933 byte} {2021. 03. 22. 13:48:36} {}
\fajl{SSMoveTest.java}{1384 byte} {2021. 03. 20. 15:46:46} {}
\fajl{SSPlaceItemTest.java}{2044 byte} {2021. 03. 22. 13:50:20} {}

\end{fajllista}

\subsection{Fordítás}
%\comment{A fenti listában szereplő forrásfájlokból milyen műveletekkel lehet a bináris, futtatható kódot előállítani. Az előállításhoz csak a 2. Követelmények c. dokumentumban leírt környezetet szabad előírni.}

A fenti fájlokból a javac parancs kiadásával fordítható a tesztelőprogram (az alábbi módon).
Esetleg GUI-val rendelkező fejlesztőkörnyezetből, az annak megfelelő módon.

\begin{verbatim}
    javac -d bin *.java
\end{verbatim}

\subsection{Futtatás}
%\comment{A futtatható kód elindításával kapcsolatos teendők leírása. Az indításhoz csak a 2. Követelmények c. dokumentumban leírt környezetet szabad előírni.}

Futtatáshoz a következő parancsokat kell kiadni.
\begin{verbatim}
    cd bin
    java SkeletonEntry
\end{verbatim}

\clearpage
\section{Értékelés}
%\comment{A projekt kezdete óta az értékelésig eltelt időben tagokra bontva, százalékban.}
\begin{ertekeles}
	  \ertekelestag{Dömötör Péter}{ G2Y5TI }{25\%}
	  \ertekelestag{Sike Ádám}{ E8Z277 }{20\%}
	  \ertekelestag{Gao Tong}{ I2SVOS }{20\%}
	  \ertekelestag{Nagy Beáta}{ GPOGC5 }{20\%}
	  \ertekelestag{Tatai Titusz Miklós}{ IJHLYX }{15\%}
\end{ertekeles}



\section{Napló}

\begin{naplo}
	\naplotag{2021.03.17. 10:00 }{ 2 óra }{ Csapat }
	{ 
		Konzultáció és előző heti munkánk értékelése, feladatok kiosztása   
	}

	\naplotag{2021.03.20. 10:00 }{ 2 óra }{ Csapat }
	{ 
		Az addig elkészített mukát megbeszélése.   
	}

	\naplotag{2021.03.20. 14:00 }{ 2 óra }{ Gao Tong }
	{ 
		Drill, Hide tesztelés leírása.   
	}

	\naplotag{2021.03.21. 12:00 }{ 2 óra }{ Gao Tong }
	{ 
		Hide, PlaceResurce tesztelés leírása.   
	}

	\naplotag{2021.03.21. 12:00 }{ 1 óra }{ Gao Tong }
	{ 
		Tesztelés javítása.
	}

	\naplotag{2021.03.21. 16:00 }{ 1 óra }{ Sike Ádám }
	{ 
		SolarFlare-es és MoveAField elkészítése
	}

	\naplotag{2021.03.21. 19:00 }{ 1 óra }{ Nagy Beáta }
	{ 
		Craft-ok tesztjei
	}

	\naplotag{2021.03.22. 10:00 }{ 1 óra }{ Sike Ádám }
	{ 
		Aprób javítások.
	}

	\naplotag{2021.03.22. 12:00 }{ 1 óra }{ Dömötör Péter }
	{ 
		Teszek készítése.
	}

	\naplotag{2021.03.17. 10:00 }{ 2 óra }{ Csapat }
	{ 
		Konzultáció és előző heti munkánk értékelése, feladatok kiosztása   
	}
\end{naplo}


\end{document}