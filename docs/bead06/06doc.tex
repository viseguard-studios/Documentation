\documentclass[../../projlab]{subfiles}
\begin{document}

\makeatletter

\ifSubfilesClassLoaded{
	\coverpage{5. Leadás}
	\def\filePath[#1]{./../../#1}
}{}

\makeatother

\chapter{Szkeleton beadás}

\section{Fordítási és futtatási útmutató}
\comment{A feltöltött program fordításával és futtatásával kapcsolatos útmutatás. Ennek tartalmaznia kell leltárszerűen az egyes fájlok pontos nevét, méretét byte-ban, keletkezési idejét, valamint azt, hogy a fájlban mi került megvalósításra.}

\subsection{Fájllista}

\begin{fajllista}
	\fajl{Asteroid.java}{10214 byte} {2021. 03. 22. 13:50:20} {Aszteroidákért felelős osztály}
\fajl{Engine.java}{615 byte} {2021. 03. 20. 13:48:06} {A játék enginért felelős osztály}
\fajl{Entity.java}{916 byte} {2021. 03. 22. 13:48:36} {Az absztrakt, entitásokért felelős osztály}
\fajl{GameManager.java}{3703 byte} {2021. 03. 22. 12:20:42} {A játék managerének osztálya}
\fajl{Inventory.java}{3992 byte} {2021. 03. 22. 13:48:36} {Raktárokért felelős osztály}
\fajl{Main.java}{159 byte} {2021. 03. 20. 13:48:07} {A program indítását végző osztály}
\fajl{Player.java}{1069 byte} {2021. 03. 22. 12:20:42} {Egy-egy játékosért felelős osztály}
\fajl{Robot.java}{1232 byte} {2021. 03. 22. 13:45:56} {Robotokért felelős osztály}
\fajl{Scene.java}{1848 byte} {2021. 03. 22. 13:48:36} {Játéktérért felelős osztály}
\fajl{SpaceShip.java}{3897 byte} {2021. 03. 22. 13:17:36} {Telepesekért felelős osztály}
\fajl{Vessel.java}{5116 byte} {2021. 03. 22. 13:45:55} {Járművekért felelős osztály}
\fajl{Logger.java}{1916 byte} {2021. 03. 22. 10:22:18} {Általános log-oló függvények osztálya}
\fajl{Registry.java}{435 byte} {2021. 03. 20. 14:11:05} {Tesztfüggvények tárolását végző osztály}
\fajl{SkeletonEntry.java}{1860 byte} {2021. 03. 22. 13:48:36} {Tesztek futtatásáért felelős osztály}
\fajl{STest.java}{212 byte} {2021. 03. 20. 13:35:59} {Annotáció (nem lett használva) }
\fajl{Test.java}{203 byte} {2021. 03. 20. 13:48:07} {Általános tesztosztály}
\fajl{Building.java}{1121 byte} {2021. 03. 20. 13:48:06} {Általános épülettípus osztálya}
\fajl{BuildingPlace.java}{124 byte} {2021. 03. 20. 13:48:06} {Enum az épület lehetséges helyére egy aszteroidán belül}
\fajl{SpaceStation.java}{2037 byte} {2021. 03. 22. 10:11:15} {Felépített űrállomás osztálya}
\fajl{TeleportGate.java}{1119 byte} {2021. 03. 20. 13:48:06} {Lerakott teleportkapu-pár osztálya}
\fajl{Coal.java}{1003 byte} {2021. 03. 22. 11:59:05} {Kibányászott Szén nyersanyag osztálya}
\fajl{Ice.java}{1002 byte} {2021. 03. 22. 11:59:05} {Kibányászott Vízjég nyersanyag osztálya}
\fajl{Iron.java}{1005 byte} {2021. 03. 22. 11:59:05} {Kibányászott Vas nyersanyag osztálya}
\fajl{Item.java}{1583 byte} {2021. 03. 22. 11:59:05} {Általános kibányászott nyersanyag osztálya}
\fajl{TeleportGateItem.java}{1314 byte} {2021. 03. 22. 11:59:05} {Teleportkapu elem osztálya}
\fajl{Titan.java}{1011 byte} {2021. 03. 22. 11:59:05} {Kibányászott Titán nyersanyag osztálya}
\fajl{Uranium.java}{1023 byte} {2021. 03. 22. 11:59:05} {Kibányászott Urán nyersanyag osztálya}
\fajl{Recipe.java}{5733 byte} {2021. 03. 22. 13:21:00} {Általános receptért felelős osztály}
\fajl{RobotRecipe.java}{1274 byte} {2021. 03. 22. 13:19:29} {Robot-recept osztálya}
\fajl{SpaceStationRecipe.java}{1303 byte} {2021. 03. 22. 13:19:29} {Űrállomás-recept osztálya}
\fajl{TeleportGateRecipe.java}{5566 byte} {2021. 03. 22. 13:21:00} {Teleportkapu-recept osztálya}
\fajl{CoalResource.java}{785 byte} {2021. 03. 22. 10:19:53} {Szén nyersanyag osztálya}
\fajl{IceResource.java}{780 byte} {2021. 03. 22. 10:19:53} {Vízjég nyersanyag osztálya}
\fajl{IronResource.java}{782 byte} {2021. 03. 22. 10:19:53} {Vas nyersanyag osztálya}
\fajl{Resource.java}{1493 byte} {2021. 03. 22. 10:08:14} {Egy általános nyersanyagért felelős osztály}
\fajl{TitaniumResource.java}{795 byte} {2021. 03. 22. 10:19:53} {Titán nyersanyag osztálya}
\fajl{UraniumResource.java}{798 byte} {2021. 03. 22. 10:19:53} {Urán nyersanyag osztálya}
\fajl{CraftRobotTest.java}{2875 byte} {2021. 03. 22. 13:19:29} {Robotok készítésének tesztelése}
\fajl{CraftSpaceStationTest.java}{3218 byte} {2021. 03. 22. 13:48:36} {Űrállomás építésének tesztelése}
\fajl{CraftTeleportGateTest.java}{3139 byte} {2021. 03. 22. 13:19:29} {Teleportkapu-pár építésének tesztelése}
\fajl{MoveAsteroidField.java}{980 byte} {2021. 03. 22. 12:20:42} {Aszteroidamező mozgatásának tesztelése}
\fajl{QueueSolarFlareTest.java}{828 byte} {2021. 03. 22. 12:20:42} {Napkitörés tesztosztálya}
\fajl{SolarFlareHitsRobot.java}{965 byte} {2021. 03. 22. 13:17:36} {Napkitörés eléri a robotot tesztosztály}
\fajl{SolarFlareHitsSpaceShip.java}{985 byte} {2021. 03. 22. 13:17:36} {Napkitörés eléri a telepest tesztosztály}
\fajl{SSDrillTest.java}{1119 byte} {2021. 03. 22. 10:08:14} {Telepesek aszteroidát fúrásának tesztosztálya}
\fajl{SSExitHidingTest.java}{948 byte} {2021. 03. 22. 10:08:14} {Járművek aszteroidából kibújásának tesztosztálya}
\fajl{SSHideTest.java}{1933 byte} {2021. 03. 22. 13:48:36} {Járművek aszteroidába bújásának tesztosztálya}
\fajl{SSMoveTest.java}{1384 byte} {2021. 03. 20. 15:46:46} {Telepesek mozgásának tesztosztálya}
\fajl{SSPlaceItemTest.java}{2044 byte} {2021. 03. 22. 13:50:20} {Elemek raktárba helyezésének tesztosztálya}

\end{fajllista}

\subsection{Fordítás}
\comment{A fenti listában szereplő forrásfájlokból milyen műveletekkel lehet a bináris, futtatható kódot előállítani. Az előállításhoz csak a 2. Követelmények c. dokumentumban leírt környezetet szabad előírni.}

A fenti fájlokból a javac parancs kiadásával fordítható a tesztelőprogram (az alábbi módon).
Esetleg GUI-val rendelkező fejlesztőkörnyezetből, az annak megfelelő módon.

\begin{verbatim}
    javac -d bin *.java
\end{verbatim}

\subsection{Futtatás}
\comment{A futtatható kód elindításával kapcsolatos teendők leírása. Az indításhoz csak a 2. Követelmények c. dokumentumban leírt környezetet szabad előírni.}

Futtatáshoz a következő parancsokat kell kiadni.
\begin{verbatim}
    cd bin
    java SkeletonEntry
\end{verbatim}

\clearpage
\section{Értékelés}
\comment{A projekt kezdete óta az értékelésig eltelt időben tagokra bontva, százalékban.}
\begin{ertekeles}
	  \ertekelestag{Teszt János}{ ?????? }{100\%}
\end{ertekeles}


\end{document}