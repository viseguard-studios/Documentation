\documentclass[../../projlab]{subfiles}
\begin{document}

\makeatletter

\ifSubfilesClassLoaded{
	\coverpage{13. Leadás}
	%\renewcommand{\filePath}[1]{./../../ #1}
	\def\filePath[#1]{./../../#1}
}{}

\makeatother


\chapter{Grafikus felület specifikációja}

\section{Fordítási és futtatási útmutató}
\comment{A feltöltött program fordításával és futtatásával kapcsolatos útmutatás. Ennek tartalmaznia kell leltárszerűen az egyes fájlok pontos nevét, méretét byte-ban, keletkezési idejét, valamint azt, hogy a fájlban mi került megvalósításra.}

\subsection{Fájllista}

\begin{fajllista}
    
\end{fajllista}

\subsection{Fordítás}
A fenti fájlokból az álltalunk készített compile.bat segítségével lehet jar file-t készíteni.

\begin{verbatim}
    .\compile.bat
\end{verbatim}

\subsection{Futtatás}

Futtatáshoz a következő parancsokat kell kiadni.
\begin{verbatim}
    java -jar .\AsteroidMiner.jar
\end{verbatim}

\clearpage
\section{Értékelés}
\comment{A projekt kezdete óta az értékelésig eltelt időben tagokra bontva, százalékban.}
\begin{ertekeles}
    \ertekelestag{Dömötör Péter}{ G2Y5TI }{\%}
    \ertekelestag{Sike Ádám}{ E8Z277 }{\%}
    \ertekelestag{Gao Tong}{ I2SVOS }{\%}
    \ertekelestag{Nagy Beáta}{ GPOGC5 }{\%}
    \ertekelestag{Tatai Titusz Miklós}{ IJHLYX }{\%}
\end{ertekeles}


\section{Napló}

\begin{naplo}
    \naplotag{feb. 18. 16h }{ 1 óra }{ Csapat }{ Értekezlet  \newline Döntés: Segédeszközök kiválasztása (git, trello, drive)}
\end{naplo}

\end{document}