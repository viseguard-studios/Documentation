\subsection{Asteroid}
\begin{class-template-responsibility}
Egy aszteroidát jelöl. Minden példányának létrejöttekor beállítódnak a generátor álltal a tulajdonságai.
\end{class-template-responsibility}
\begin{class-template-interface}
$\emptyset$
\end{class-template-interface}
\begin{class-template-baseclass}
Entity
\end{class-template-baseclass}
\begin{class-template-attribute}
\classitem{-coreSize:int}{Az aszteroida magjának mérete, ennyi egységnyi nyersanyag bányászható ki belőle a játék elején, és bányászat után ennyi egységnyi item helyezhető vissza bele. }
\classitem{-exploded:bool}{Igaz, ha felrobbant már az aszteroida, hamis, ha nem. }
\classitem{-maxHidingSpace:int}{Maximum ennyi telepes számára van az adott időpillanatban hely az aszteroidában elbújni. Minden telepes sikeres elbújása után csökken ez a szám. }
\classitem{-crustSize:int}{Az aszteroida köpenyének mélysége. Ennyi egységet kell még fúrni benne jelenleg, hogy teljesen átfúrt legyen. Minden fúrás után csökken ez a szám. }
\classitem{-revealed:bool}{Igaz, ha az aszteroida már „fel van fedve” a térképen, tehát rajta vagy bármelyik szomszédján már járt telepes vagy robot. Ellenkező esetben hamis.}
\classitem{-visited:bool}{Igaz, ha az aszteroidán már járt telepes vagy robot, ellenkező esetben hamis.}
\classitem{-hidingVessel:Vessel}{Ha az aszteroidában jelenleg megbújik egy telepes, akkor eltárolja, melyik telepesről van szó. Ha nem bújik benne senki, null értéket tárol.}
\classitem{-inventory:Inventory[1]}{Az aszteroida raktára, ami a magba belehelyezett és jelenleg ott tárolt elemeket tartalmazza. }
\classitem{-neighbours:Asteroid[1..*]}{A szomszédos aszteroidák tárolója.}
\classitem{-resource:Resource[0..1]}{Az aszteroida magjában lévő nyersanyagkészlet. }
\classitem{-stationed:Vessel[0..*]}{Az aszteroidán jelenleg állomásozó járművek tárolója. }
\classitem{-buildings:Building[0..*]}{Az aszteroidára épített (véglegesen elhelyezett) építmények tárolója. }
\end{class-template-attribute}
\begin{class-template-method}
\classitem{+Asteroid(r:Resource):Asteroid}{}
\classitem{+Explode()}{Felrobban az aszteroida. Felrobbantja az összes rajta tartózkodó járművet, hozzáférhetetlenné teszi a raktárat és a rajta lévő épületeket, értesíti a szomszédos aszteroidákat a robbanásról. }
\classitem{+ReachableAsteroids():Asteroid}{Tárolja, hogy melyik aszteroidák elérhetőek jelenleg az adott aszteroidából. }
\classitem{+GetResourceType():}{Visszaadja az aszteroida magjában található nyersanyag típusát. }
\classitem{+Depart(v:Vessel)}{Egy adott jármű elhagyja az aszteroidát, törlődik az ott tartózkodók közül. }
\classitem{+Arrive(v:Vessel)}{Egy adott jármű érkezik az aszteroidára, regisztrálja az ott tartózkodók közé.}
\classitem{+GetAvailableHidingSpace():int}{Visszaadja, hogy hány telepes számára van még hely elbújásra jelenleg az aszteroidában, attól függően, hogy jelentleg elbújt-e telepes az aszteroidában. }
\classitem{+Reveal()}{Aszteroida felfedése a térképen, amennyiben még nem volt felfedve.  }
\classitem{+Drill()}{Akkor hívódik meg amikor az aszteroida kérgén lévő jukat akarják mélyíteni.}
\classitem{+Mine():Item}{Akkor hívódik meg amikor az aszteroidából akarnak nyersanyagok kibányászni.}
\classitem{+Hide(v:Vessel):bool}{Az aszteroidában megpróbál elbújni egy jámű. Igazat ad vissza ha el tud bújni, hamisat ha nem.}
\classitem{+Exit(v:Vessel)}{Az adott jámű előbújik az aszteroidából.}
\classitem{+AddBuilding(b:Building)}{Hozzáad egy új épületet az aszteroidához.}
\classitem{+PlaceItem(i:Item):bool}{Elem belehelyezése a raktárba. Sikeres behelyezés után igazzal tér vissza, sikertelen művelet után hamissal tér vissza. }
\classitem{+AddNeigbour()}{}
\classitem{+GetInventory():return:Inventory}{}
\end{class-template-method}


\subsection{Building}
\begin{class-template-responsibility}
Egy adott konkrét épületet reprezentál, a type mezőben tárolja el, hogy mejik konkrét épület típus.
\end{class-template-responsibility}
\begin{class-template-interface}
$\emptyset$
\end{class-template-interface}
\begin{class-template-baseclass}
$\emptyset$
\end{class-template-baseclass}
\begin{class-template-attribute}
\classitem{-asteroid:Asteroid[1]}{}
\classitem{-position:BuildingPlace}{Azt jelöli hogy az épület hol helyezkedik el az aszteroidán.}
\end{class-template-attribute}
\begin{class-template-method}
\classitem{+Explode()}{akkor hívódik meg ha valamijen okból az épület megsemmisül.}
\classitem{+GetRoutes():Asteroid}{Az ebből az épületből elérhető extra aszteroidákat adja vissza.}
\classitem{+Building(a:Asteroid):Building}{}
\end{class-template-method}


\subsection{Coal}
\begin{class-template-responsibility}

\end{class-template-responsibility}
\begin{class-template-interface}
$\emptyset$
\end{class-template-interface}
\begin{class-template-baseclass}
Item
\end{class-template-baseclass}
\begin{class-template-attribute}
\item[] $\emptyset$
\end{class-template-attribute}
\begin{class-template-method}
\classitem{+Satisfies(i:Item):bool}{}
\end{class-template-method}


\subsection{CoalResource}
\begin{class-template-responsibility}

\end{class-template-responsibility}
\begin{class-template-interface}
$\emptyset$
\end{class-template-interface}
\begin{class-template-baseclass}
Resource
\end{class-template-baseclass}
\begin{class-template-attribute}
\item[] $\emptyset$
\end{class-template-attribute}
\begin{class-template-method}
\classitem{+Satisfies(r:Resource):bool}{}
\end{class-template-method}


\subsection{Engine}
\begin{class-template-responsibility}
A program indításáért, egy játékfolyam bezárásáért felelős osztály. 
\end{class-template-responsibility}
\begin{class-template-interface}
$\emptyset$
\end{class-template-interface}
\begin{class-template-baseclass}
$\emptyset$
\end{class-template-baseclass}
\begin{class-template-attribute}
\item[] $\emptyset$
\end{class-template-attribute}
\begin{class-template-method}
\classitem{+StartApplication()}{A program indítása.}
\classitem{+StartGame()}{Egy új játék kezdése.}
\classitem{+EndGame()}{Játékmenetek befejezése. }
\end{class-template-method}


\subsection{Entity}
\begin{class-template-responsibility}
Egy adott entitás (pozícióval, vizuális megjelenítéssel rendelkező játékelem) osztálya. 
\end{class-template-responsibility}
\begin{class-template-interface}
$\emptyset$
\end{class-template-interface}
\begin{class-template-baseclass}
$\emptyset$
\end{class-template-baseclass}
\begin{class-template-attribute}
\classitem{-position:}{Az aktuális pozíció, csak belülről állítható.}
\classitem{Not definedscene:Scene[1]}{A játékban szereplő összes entitás tárolója. }
\end{class-template-attribute}
\begin{class-template-method}
\classitem{+Update()}{}
\classitem{+RoundEnd()}{Akkor hívódik meg a az adott körben már minden játékos lépett. A robotot ezt haszmálják például a mozgásra.}
\classitem{+SetPosition()}{}
\classitem{+GetScene():Scene}{}
\end{class-template-method}


\subsection{GameManager}
\begin{class-template-responsibility}
A játék menetének irányításáért felelős osztály. 
\end{class-template-responsibility}
\begin{class-template-interface}
$\emptyset$
\end{class-template-interface}
\begin{class-template-baseclass}
$\emptyset$
\end{class-template-baseclass}
\begin{class-template-attribute}
\classitem{-currentPlayer:}{The player who is taking the turn currently}
\classitem{-selectedVessel:Vessel}{Az aktuálisan irányítható jármű (akit a soron lévő játékos jelenleg irányít). }
\classitem{-sunDistance:int}{A Naptól való aktuális távolság. }
\classitem{-allPlayers:Player[1..*]}{A játék játékosainak listája. }
\classitem{-asteroids:Asteroid[1..*]}{}
\classitem{-gameEnded:bool}{}
\classitem{-settlers:SpaceShip}{A játék telepeseinek tárolója. }
\classitem{-currentPlayer:Player[1]}{A jelenleg aktív játékos, aki éppen tudja mozgatni a telepesit.}
\classitem{-scene:Scene}{}
\end{class-template-attribute}
\begin{class-template-method}
\classitem{+InitGame()}{}
\classitem{+AddPlayer(p:Player)}{}
\classitem{+StartGame()}{Új játék indítása, a játékosok listájának felvétele, járművek pozíciójának és a Naptól való távolságnak az inicializálása. }
\classitem{+TakeTurn()}{Egy játékos aktuális köre - ekkor van lehetősége irányítani a járműveit egyesével. }
\classitem{+EndGame()}{Aktuális játék befejezése. }
\classitem{-GenerateScene()}{}
\classitem{-GenerateNewResource():Resource}{}
\classitem{-GenerateAsteroids()}{}
\classitem{+IsSunStormActive()}{}
\end{class-template-method}


\subsection{Ice}
\begin{class-template-responsibility}

\end{class-template-responsibility}
\begin{class-template-interface}
$\emptyset$
\end{class-template-interface}
\begin{class-template-baseclass}
Item
\end{class-template-baseclass}
\begin{class-template-attribute}
\item[] $\emptyset$
\end{class-template-attribute}
\begin{class-template-method}
\classitem{+Satisfies(i:Item):bool}{}
\end{class-template-method}


\subsection{IceResource}
\begin{class-template-responsibility}

\end{class-template-responsibility}
\begin{class-template-interface}
$\emptyset$
\end{class-template-interface}
\begin{class-template-baseclass}
Resource
\end{class-template-baseclass}
\begin{class-template-attribute}
\item[] $\emptyset$
\end{class-template-attribute}
\begin{class-template-method}
\classitem{+Satisfies(r:Resource):bool}{}
\classitem{+NearSun()}{}
\end{class-template-method}


\subsection{Inventory}
\begin{class-template-responsibility}
Egy aszteroidán vagy telepesnél található kinyert nyersanyagok (itemek) tárolója.
Ezt az osztályt használja az Aszteroida illetve a telepesk ürhejója a megszerzett nyersanyagok tárolására.
\end{class-template-responsibility}
\begin{class-template-interface}
$\emptyset$
\end{class-template-interface}
\begin{class-template-baseclass}
$\emptyset$
\end{class-template-baseclass}
\begin{class-template-attribute}
\classitem{-size:int}{A tároló kapacitása. }
\classitem{-items:Item[0..*]}{A tárolóban aktuálisan tárolt elemek. }
\end{class-template-attribute}
\begin{class-template-method}
\classitem{+InsertItem(item:Item):bool}{Új elem hozzáadása a tárolóhoz, amennyiben van benne szabad hely (kapacitás).  Sikeres művelet esetén igaz, sikertelen művelet esetén hamis visszatérési értéke van. }
\classitem{+RemoveItem(item:Item)}{Elem eltávolítása a tárolóból. }
\classitem{+TryInsertItem(item:Item):bool}{Ellenőrzi, hogy az adott elem elméletileg belehelyezhető-e a raktárba. }
\end{class-template-method}


\subsection{Iron}
\begin{class-template-responsibility}

\end{class-template-responsibility}
\begin{class-template-interface}
$\emptyset$
\end{class-template-interface}
\begin{class-template-baseclass}
Item
\end{class-template-baseclass}
\begin{class-template-attribute}
\item[] $\emptyset$
\end{class-template-attribute}
\begin{class-template-method}
\classitem{+Satisfies(i:Item):bool}{}
\end{class-template-method}


\subsection{IronResource}
\begin{class-template-responsibility}

\end{class-template-responsibility}
\begin{class-template-interface}
$\emptyset$
\end{class-template-interface}
\begin{class-template-baseclass}
Resource
\end{class-template-baseclass}
\begin{class-template-attribute}
\item[] $\emptyset$
\end{class-template-attribute}
\begin{class-template-method}
\classitem{+Satisfies(r:Resource):bool}{}
\end{class-template-method}


\subsection{Item}
\begin{class-template-responsibility}
A már kibányászott nyersanyagok hordozható elemekké válnak, ebben a formában tárolja őket az osztály. Egy típusú, adott mennyiségű azonos elemet, illetve hordozható épületet (pl. teleportkapu-pár) tárol.
\end{class-template-responsibility}
\begin{class-template-interface}
$\emptyset$
\end{class-template-interface}
\begin{class-template-baseclass}
$\emptyset$
\end{class-template-baseclass}
\begin{class-template-attribute}
\classitem{-amount:int}{Az elem darabszámát tárolja. }
\end{class-template-attribute}
\begin{class-template-method}
\classitem{+Satisfies(i:Item):bool}{}
\classitem{+Reduce(a:int)}{}
\end{class-template-method}


\subsection{Player}
\begin{class-template-responsibility}
Egy adott játékos reprezentációja. 
\end{class-template-responsibility}
\begin{class-template-interface}
$\emptyset$
\end{class-template-interface}
\begin{class-template-baseclass}
$\emptyset$
\end{class-template-baseclass}
\begin{class-template-attribute}
\classitem{-name:string}{A játékos neve.}
\classitem{-searching\_for:Resource[1]}{}
\end{class-template-attribute}
\begin{class-template-method}
\item[] $\emptyset$
\end{class-template-method}


\subsection{Recipe}
\begin{class-template-responsibility}
A telepesek által elkészíthető receptekért felel. Egy recept adott számú kibányászott nyersanyag (elem) felhasználásával elkészített dolgok pontos hozzávalóit, és az elkészült eredmény típusát tárolja. 
\end{class-template-responsibility}
\begin{class-template-interface}
$\emptyset$
\end{class-template-interface}
\begin{class-template-baseclass}
$\emptyset$
\end{class-template-baseclass}
\begin{class-template-attribute}
\classitem{-input:Item[1..*]}{Az aktuális recepthez szükséges hozzávalók tárolása. }
\end{class-template-attribute}
\begin{class-template-method}
\classitem{+CanCraft(inv:Inventory, a:Asteroid):bool}{Meghatározza, hogy egy adott raktárban és aszteroidán lévő kibányászott nyersanyagkészlet elegendő-e az adott recept elkészítéséhez. }
\classitem{+Craft(inv:Inventory, a:Asteroid)}{Akkor hívódik meg ha ténylegesen le akarjuk gyártani ezt a receptet.}
\classitem{Not definedMakeResult(inv:Inventory, a:Asteroid)}{}
\end{class-template-method}


\subsection{Resource}
\begin{class-template-responsibility}
Egy adott aszteroidában tárolt egy fajta (egy konkrét típusú), adott számú egységgel rendelkező, bányászással kinyerhető nyersanyagkészletért felel. 
\end{class-template-responsibility}
\begin{class-template-interface}
$\emptyset$
\end{class-template-interface}
\begin{class-template-baseclass}
$\emptyset$
\end{class-template-baseclass}
\begin{class-template-attribute}
\classitem{-amount:int}{A játék adott pillanatában ennyi egységnyi nyersanyag bányászható még ki az aszteroidából. }
\end{class-template-attribute}
\begin{class-template-method}
\classitem{+Mine():Item}{Egy egység nyersanyag kinyerése bányászattal, amennyiben ez lehetséges (rendelkezésre áll elég nyersanyag). Ekkor egy kibányászott megfelelő elemmel tér vissza.}
\classitem{+NearSun()}{Napközelben a nyersanyag típusának megfelelő műveletet hajt végre. Különleges képességekkel nem rendelkező nyersanyagok esetén nem hajt végre műveletet.  }
\classitem{+Satisfies(r:Resource):bool}{}
\end{class-template-method}


\subsection{Robot}
\begin{class-template-responsibility}
Egy speciális jármű, a robot tevékenységeit, tulajdonságait tartalmazza.
\end{class-template-responsibility}
\begin{class-template-interface}
$\emptyset$
\end{class-template-interface}
\begin{class-template-baseclass}
Entity \baseclass Vessel
\end{class-template-baseclass}
\begin{class-template-attribute}
\item[] $\emptyset$
\end{class-template-attribute}
\begin{class-template-method}
\classitem{+Robot(p:Player, a:Asteroid):Robot}{}
\classitem{+GetHidingSpaceRequirement():int}{A robotok által az aszteroida magjában elfoglalt hely nagyságát adja vissza. Korlátlan mennyiségű robot elfér, tehát 0 az értéke. }
\end{class-template-method}


\subsection{RobotRecipe}
\begin{class-template-responsibility}
Robotok legyártásáért feleős osztály.
\end{class-template-responsibility}
\begin{class-template-interface}
$\emptyset$
\end{class-template-interface}
\begin{class-template-baseclass}
Recipe
\end{class-template-baseclass}
\begin{class-template-attribute}
\item[] $\emptyset$
\end{class-template-attribute}
\begin{class-template-method}
\item[] $\emptyset$
\end{class-template-method}


\subsection{Scene}
\begin{class-template-responsibility}
Az aktuális játék típusától független, játékmenetért felelős osztály. 
\end{class-template-responsibility}
\begin{class-template-interface}
$\emptyset$
\end{class-template-interface}
\begin{class-template-baseclass}
$\emptyset$
\end{class-template-baseclass}
\begin{class-template-attribute}
\classitem{Not definedentities:Entity[1..*]}{}
\classitem{-manager:GameManager}{}
\end{class-template-attribute}
\begin{class-template-method}
\classitem{+AddEntity(e:Entity)}{}
\classitem{+RoundEnded()}{}
\classitem{+GetManager():GameManager}{}
\end{class-template-method}


\subsection{SpaceShip}
\begin{class-template-responsibility}
A telepesekért felelős osztály.
\end{class-template-responsibility}
\begin{class-template-interface}
$\emptyset$
\end{class-template-interface}
\begin{class-template-baseclass}
Entity \baseclass Vessel
\end{class-template-baseclass}
\begin{class-template-attribute}
\classitem{-inventory:Inventory[1]}{A telepes által folyamatosan hordozott raktár. }
\end{class-template-attribute}
\begin{class-template-method}
\classitem{+SpaceShip(p:Player, a:Asteroid):SpaceShip}{}
\classitem{+GetHidingSpaceRequirement():int}{A telepesek által az aszteroida magjában elfoglalt hely nagyságát adja vissza. }
\classitem{+Mine()}{Teljesen átfúrt, nem üres magú aszteroidán való tartózkodás esetén a telepes egy egységet kibányászik az ott található nyersanyagból. Az így keletkező elem a telepes raktárába kerül.  Ha sikertelen a művelet (nem teljesülnek a feltételek), nem történik művelet. }
\classitem{+Craft(recipe:Recipe)}{A telepes az aktuálisan rendelkezésére álló elemekből egy "receptet" készít.  A felhasznált elemek elhasználódnak, törlődnek a raktárból. }
\classitem{+PlaceItem(i:Item):bool}{Elem belehelyezése az aktuális aszteroidába, amin tartózkodik. Sikeres behelyezés után igazzal tér vissza, a saját raktárból eltávolítja az elemet. Sikertelen művelet után hamissal tér vissza. }
\end{class-template-method}


\subsection{SpaceStation}
\begin{class-template-responsibility}
A játékosok álltal megépítendő űrálomás épület típust jelöli. Speciális tulajdonsága hogy amikor megépül akkor a játék befejeződik.
\end{class-template-responsibility}
\begin{class-template-interface}
$\emptyset$
\end{class-template-interface}
\begin{class-template-baseclass}
Building
\end{class-template-baseclass}
\begin{class-template-attribute}
\item[] $\emptyset$
\end{class-template-attribute}
\begin{class-template-method}
\classitem{+SpaceStation(a:Asteroid):SpaceStation}{meghívódik ha az adott aszteroidán egy űrállomás épült. Akkor a játék befejeződik.}
\classitem{+Explode(a:Asteroid)}{}
\classitem{+GetRoutes():Asteroid}{}
\end{class-template-method}


\subsection{SpaceStationRecipe}
\begin{class-template-responsibility}
Aszteroidákra elhelyezhető épületek receptjeiért felelős osztály. 
\end{class-template-responsibility}
\begin{class-template-interface}
$\emptyset$
\end{class-template-interface}
\begin{class-template-baseclass}
Recipe
\end{class-template-baseclass}
\begin{class-template-attribute}
\item[] $\emptyset$
\end{class-template-attribute}
\begin{class-template-method}
\item[] $\emptyset$
\end{class-template-method}


\subsection{TeleportGate}
\begin{class-template-responsibility}
A teleport kapukat reprezentáló osztály.
\end{class-template-responsibility}
\begin{class-template-interface}
$\emptyset$
\end{class-template-interface}
\begin{class-template-baseclass}
Building
\end{class-template-baseclass}
\begin{class-template-attribute}
\item[] $\emptyset$
\end{class-template-attribute}
\begin{class-template-method}
\classitem{+GetRoutes():Asteroid}{Visszadja az ebből a kapuból elérhető aszteroidákat.
}
\classitem{+PairDestroyed()}{Akkor hívódik meg ha a párja megsemmisül és ezéltal ez az oldal deaktiválódik.}
\classitem{+Explode(a:Asteroid)}{Akkor hívódik meg ha az épület felrobban. Ekkor szól a szomszédjának is a PairDestroyed() fügvénnyel.}
\end{class-template-method}


\subsection{TeleportGateItem}
\begin{class-template-responsibility}

\end{class-template-responsibility}
\begin{class-template-interface}
$\emptyset$
\end{class-template-interface}
\begin{class-template-baseclass}
Item
\end{class-template-baseclass}
\begin{class-template-attribute}
\item[] $\emptyset$
\end{class-template-attribute}
\begin{class-template-method}
\classitem{+Satisfies(i:Item):bool}{}
\end{class-template-method}


\subsection{TeleportGateRecipe}
\begin{class-template-responsibility}
Teleportkapu-párok elkészítéséhez szükséges speciális recept típus. 
\end{class-template-responsibility}
\begin{class-template-interface}
$\emptyset$
\end{class-template-interface}
\begin{class-template-baseclass}
Recipe
\end{class-template-baseclass}
\begin{class-template-attribute}
\classitem{-result:TeleportGate}{A sikeres készítés után ilyen típusú eredményhez ("elemhez") lehet hozzájutni. }
\classitem{-amount:int}{Az elkészült dolog mennyisége (egyszerre hány egység készül el belőle)}
\end{class-template-attribute}
\begin{class-template-method}
\item[] $\emptyset$
\end{class-template-method}


\subsection{Titan}
\begin{class-template-responsibility}

\end{class-template-responsibility}
\begin{class-template-interface}
$\emptyset$
\end{class-template-interface}
\begin{class-template-baseclass}
Item
\end{class-template-baseclass}
\begin{class-template-attribute}
\item[] $\emptyset$
\end{class-template-attribute}
\begin{class-template-method}
\classitem{+Satisfies(i:Item):bool}{}
\end{class-template-method}


\subsection{TitaniumResource}
\begin{class-template-responsibility}

\end{class-template-responsibility}
\begin{class-template-interface}
$\emptyset$
\end{class-template-interface}
\begin{class-template-baseclass}
Resource
\end{class-template-baseclass}
\begin{class-template-attribute}
\item[] $\emptyset$
\end{class-template-attribute}
\begin{class-template-method}
\classitem{+Satisfies(r:Resource):bool}{}
\end{class-template-method}


\subsection{Uranium}
\begin{class-template-responsibility}

\end{class-template-responsibility}
\begin{class-template-interface}
$\emptyset$
\end{class-template-interface}
\begin{class-template-baseclass}
Item
\end{class-template-baseclass}
\begin{class-template-attribute}
\item[] $\emptyset$
\end{class-template-attribute}
\begin{class-template-method}
\classitem{+Satisfies(i:Item):bool}{}
\end{class-template-method}


\subsection{UraniumResource}
\begin{class-template-responsibility}

\end{class-template-responsibility}
\begin{class-template-interface}
$\emptyset$
\end{class-template-interface}
\begin{class-template-baseclass}
Resource
\end{class-template-baseclass}
\begin{class-template-attribute}
\item[] $\emptyset$
\end{class-template-attribute}
\begin{class-template-method}
\classitem{+Satisfies(r:Resource):bool}{}
\classitem{+NearSun()}{}
\end{class-template-method}


\subsection{Vessel}
\begin{class-template-responsibility}
Egy-egy járműért (pl. telepes vagy robot) felelős osztály. 
\end{class-template-responsibility}
\begin{class-template-interface}
$\emptyset$
\end{class-template-interface}
\begin{class-template-baseclass}
Entity
\end{class-template-baseclass}
\begin{class-template-attribute}
\classitem{-isHidden:bool}{Ha jelenleg el van bújva az aszteroidájában, igaz, ellenkező esetben hamis. }
\classitem{-currentAsteroid:Asteroid[1]}{Az aktuális tartózkodási helyének (aszteroida) tárolására szolgál. }
\classitem{-owner:Player[1]}{Tárolja, hogy melyik játékos irányítja az járművet.}
\end{class-template-attribute}
\begin{class-template-method}
\classitem{+Vessel(p:Player, a:Asteroid):Vessel}{}
\classitem{+Hide()}{Belebújik az adott aszteroidába, ha van benne elegendő hely. Ha nincsen, az aszteroida felszínén marad. }
\classitem{+Drill()}{Fúr egy egységnyit az aszteroida köpenyéből, ha még nincs teljesen átfúrva. Ha át van fúrva, nem történik művelet.}
\classitem{+Move(a:Asteroid)}{Az jármű átlép egy szomszédos aszteroidára. }
\classitem{+GetHidingSpaceRequirement():int}{Absztrakt metódus, leszármazottól függően más értéket ad vissza. Meghatározza, hogy ha az adott jármű el szeretne bújni egy aszteroidában, mennyi helyre van hozzá szüksége. }
\classitem{+Explode()}{Felrobban  a jármű (egy aszteroida robbanásának hatására). }
\classitem{+ExitHiding()}{A jármű kibújik az aszteroida magjából, ha el volt bújva benne. Ha nem, nem történik művelet. }
\classitem{+AsteroidExploded()}{}
\end{class-template-method}


