\documentclass[../../projlab]{subfiles}
\begin{document}

\makeatletter

\ifSubfilesClassLoaded{
	\coverpage{4. Leadás}
	\def\filePath[#1]{./../../#1}
}{}

\makeatother

\chapter{Szkeleton tervezése}


\section{A szkeleton modell valóságos use-case-ei}

\subsection{Use-case diagram}
\begin{figure}[H] 
    \centering 
    \includegraphics[width=0.7\textwidth]{docs/Skeleton/diagrams/skeleton-use-case.eps} 
    \caption{} 
\end{figure} 

\subsection{Use-case leírások}
\begin{use-case}
    {Drill}
    {Az aszteroida fúrásának tesztelése.}
    {Tester} 
    \textbf{A} A tester hívja meg a SpaceShip-nek Drill metódust, majd a SpaceShip objektum hívja meg currentAsteroid attribútumnak Drill metódust. Ha az aszteroida már felrobban, akkor a metódus nem fog csinálni semmit. \newline
    \textbf{B} Ugyanúgy a helyzetben, ha a köpenye már teljesen át van fúrva, akkor a metódus nem csinál semmit. \newline
    \textbf{C} Ugyanúgy a helyzetben, ha a köpenye még nincs átfúrva, akkor az currentAsteroid attribútumnak crustSize-t egyel levonási értéket ad vissza. \newline
\end{use-case}

\begin{use-case}
    {Hide}
    {A telepes elbújásának tesztelése.}
    {Tester} 
    \textbf{A} A tester hívja meg a SpaceShip-nek Hide metódust. Ha a telepes éppen meg van bújva, akkor visszahívja a metódus. \newline
    \textbf{B} Ha a telepes nincs megbújva, akkor a fenti esetén még a SpaceShip objektum hívja meg currentAsteroid attribútumnak Hide metódust. Ha az aszteroida már felrobban, akkor a metódus nem fog csinálni semmit. \newline
    \textbf{C} Ugyanaz a helyzetben, ha az aszteroida köpenye még nincs átfúrva, akkor a metódus nem fog csinálni semmit. \newline
    \textbf{C} Más esetén az aszteroida Hide metódus felhívja a hidingVessel attribútumnak a GetHidingSpaceRequirement metódust, majd visszakapja egy lyukas hely mennyiségét, és megkérdezi a SpaceShip objektumnak GetHidingSpaceRequirement metódust, hogy mennyi helyet kell maradni lyukasan. Ha az első érték nagyobb mint a második, akkor a Hide metódus visszatér egy true értéket, és a SpaceShip innentől kezdődik megbújni. \newline
    \textbf{C} A tester meghívja a SpaceShip-nek a ExitHiding metódust, ha éppen a telepes nincs megbújva, akkor a metódus nem csinál semmit. \newline
    \textbf{C} Külön esetén a SpaceShip meghívja a currentAsteroid attribútumnak Hide metódust, ilyenkor kiszabadítja a aszteroida magja, és a SpaceShip végzik a megbújást. \newline
\end{use-case}

\begin{use-case}
    {PlaceResource}
    {A kinyert nyersanyag visszarakásának tesztelése.}
    {Tester} 
    \textbf{A} A tester hívja meg a SpaceShip-nek Hide metódust. Ha a telepes éppen meg van bújva, akkor visszahívja a metódus. \newline
    \textbf{B} Ha a telepes nincs megbújva, akkor a fenti esetén még a SpaceShip objektum hívja meg currentAsteroid attribútumnak Hide metódust. Ha az aszteroida már felrobban, akkor a metódus nem fog csinálni semmit. \newline
    \textbf{C} Ugyanaz a helyzetben, ha az aszteroida köpenye még nincs átfúrva, akkor a metódus nem fog csinálni semmit. \newline
    \textbf{C} Más esetén az aszteroida Hide metódus felhívja a hidingVessel attribútumnak a GetHidingSpaceRequirement metódust, majd visszakapja egy lyukas hely mennyiségét, és megkérdezi a SpaceShip objektumnak GetHidingSpaceRequirement metódust, hogy mennyi helyet kell maradni lyukasan. Ha az első érték nagyobb mint a második, akkor a Hide metódus visszatér egy true értéket, és a SpaceShip innentől kezdődik megbújni. \newline
    \textbf{C} A tester meghívja a SpaceShip-nek a ExitHiding metódust, ha éppen a telepes nincs megbújva, akkor a metódus nem csinál semmit. \newline
    \textbf{C} Külön esetén a SpaceShip meghívja a currentAsteroid attribútumnak Hide metódust, ilyenkor kiszabadítja a aszteroida magja, és a SpaceShip végzik a megbújást. \newline
\end{use-case}

\section{A szkeleton kezelői felületének terve, dialógusok}
\comment{A szkeleton által elfogadott bemenetek , valamint a szöveges konzolon megjelenő kimenetek. A kiemenet formátuma olyan kell legyen, ami alapján a működés összevethető a korábbi szekvencia-diagramokkal.}



\comment{Konzol input/output szemléltetésre:}
\begin{verbatim}
    > move
    get item
\end{verbatim}

\section{Szekvencia diagramok a belső működésre}
%\comment{A szkeletonban implementált szekvenciadiagramok. Tipikusan egy use-case egy diagram. Ezek megegyezhetnek a korábban specifikált diagramokkal, de az egyes életvonalakat (lifeline) egyértelműen a szkeletonban példányosított objektumokhoz kell tudni kötni. Azt kell megjeleníteni, hogy a szkeletonban létrehozott objektumok egymással hogyan fognak kommunikálni.}


\subsection{Controll Robots}

\begin{figure}[H] 
    \centering 
    \includegraphics[width=0.5\textwidth]{docs/Skeleton/diagrams/Controll-Robots/Move-Robots/Move-Robot-to-Asteroid.eps} 
    \caption{} 
\end{figure} 

\begin{figure}[H] 
    \centering 
    \includegraphics[width=0.5\textwidth]{docs/Skeleton/diagrams/Controll-Robots/Move-Robots/Move-Robot-Through-TG.eps} 
    \caption{} 
\end{figure} 


\begin{figure}[H] 
    \centering 
    \includegraphics[width=0.5\textwidth]{docs/Skeleton/diagrams/Controll-Robots/Robot-Drill/Robot-Drills-Asteroid-Succes.eps} 
    \caption{} 
\end{figure} 

\begin{figure}[H] 
    \centering 
    \includegraphics[width=0.5\textwidth]{docs/Skeleton/diagrams/Controll-Robots/Robot-Drill/Robot-Drills-Asteroid-Exploded.eps} 
    \caption{} 
\end{figure} 



\subsection{Drill}

\begin{figure}[H] 
    \centering 
    \includegraphics[width=1\textwidth]{docs/Skeleton/diagrams/Drill/SpaceShip-drill-succeed.eps} 
    \caption{} 
\end{figure} 

\begin{figure}[H] 
    \centering 
    \includegraphics[width=1\textwidth]{docs/Skeleton/diagrams/Drill/SpaceShip-drill-when-asteroid-exploded.eps} 
    \caption{} 
\end{figure} 

\begin{figure}[H] 
    \centering 
    \includegraphics[width=1\textwidth]{docs/Skeleton/diagrams/Drill/SpaceShip-drill-when-core-is-visible.eps} 
    \caption{} 
\end{figure} 

\subsection{Move Asteroid Field}

\begin{figure}[H] 
    \centering 
    \includegraphics[width=1\textwidth]{docs/Skeleton/diagrams/Move-Asteroid-Field/Solar-Flare-Happen/Solar-Flare-Happens.eps} 
    \caption{} 
\end{figure} 

\subsection{Place Gate}

\begin{figure}[H] 
    \centering 
    \includegraphics[width=1\textwidth]{docs/Skeleton/diagrams/Place-Gate/Place-first-gate.eps} 
    \caption{} 
\end{figure} 

\begin{figure}[H] 
    \centering 
    \includegraphics[width=1\textwidth]{docs/Skeleton/diagrams/Place-Gate/Place-second-gate.eps} 
    \caption{} 
\end{figure} 



\subsection{Place Resource}

\begin{figure}[H] 
    \centering 
    \includegraphics[width=1\textwidth]{docs/Skeleton/diagrams/Place-Resource/Asteroid-Exploded/Place-Resource-Asteroid-Exploded.eps} 
    \caption{} 
\end{figure} 

\begin{figure}[H] 
    \centering 
    \includegraphics[width=1\textwidth]{docs/Skeleton/diagrams/Place-Resource/Not-Drilled/PlaceResource-NotDrilled.eps} 
    \caption{} 
\end{figure} 

\section{Kommunikációs diagramok}
\comment{A szkeletonban, az egyes szkeleton-use-case-ek futása során létrehozott objektumok és kapcsolataik bemutatására szolgáló diagramok. Ezek alapján valósítják meg a szkeleton fejlesztői az inicializáló kódrészleteket.}

\subsection{Solar Flare}

\begin{figure}[H] 
    \centering 
    \includegraphics[width=1\textwidth]{docs/Skeleton/diagrams/Solar-Flare/Solar-Flare-Happen/Solar-Flare-Happens-Robot-Comm.eps} 
    \caption{A kör véget érésekor értesítünk minden entitást. Most történt napkitörés, ezért van a SolarFlare() meghívva.} 
\end{figure}

\begin{figure}[H] 
    \centering 
    \includegraphics[width=1\textwidth]{docs/Skeleton/diagrams/Solar-Flare/Solar-Flare-Happen/Solar-Flare-Happens-SpaceShip-Comm.eps} 
    \caption{A kör véget érésekor értesítünk minden entitást. Most történt napkitörés, ezért van a SolarFlare() meghívva.} 
\end{figure} 

\begin{figure}[H] 
    \centering 
    \includegraphics[width=1\textwidth]{docs/Skeleton/diagrams/Solar-Flare/Solar-Flare-Happen/No-Solar-Flare-Robot-Comm.eps} 
    \caption{A kör véget érésekor értesítünk minden entitást. Most nem történt napkitörés, ezért nincs a SolarFlare() meghívva.} 
\end{figure}

\begin{figure}[H] 
    \centering 
    \includegraphics[width=1\textwidth]{docs/Skeleton/diagrams/Solar-Flare/Solar-Flare-Happen/No-Solar-Flare-SpaceShip-Comm.eps} 
    \caption{A kör véget érésekor értesítünk minden entitást. Most nem történt napkitörés, ezért nincs a SolarFlare() meghívva.} 
\end{figure}


\subsection{Move Asteroid Field}

\begin{figure}[H] 
    \centering 
    \includegraphics[width=1\textwidth]{docs/Skeleton/diagrams/Move-Asteroid-Field/Change-Asteroid-Field-Distance-Comm.eps} 
    \caption{A tesztelő megváltoztatja az aszteroidamező távolságát} 
\end{figure}

\section{Napló}

\begin{naplo}

	\naplotag{2021.03.10. 10:00 }{ 2 óra }{ Csapat }
	{ 
		Konzultáció és előző heti munkánk értékelése, feladatok kiosztása   
        \newline
		Részletesen: \ref{appendix:meeting9}
	}

    
\end{naplo}

\begin{toappendix}

	\markdownInput[shiftHeadings=2]{\filePath[docs/meetings/task-breakdown-3.md]}

    \markdownInput[shiftHeadings=2]{\filePath[docs/meetings/meeting-9.md]}
	
\end{toappendix}

\end{document}