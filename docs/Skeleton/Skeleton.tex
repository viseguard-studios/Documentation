\documentclass[../../projlab]{subfiles}
\begin{document}

\makeatletter

\ifSubfilesClassLoaded{
	\coverpage{4. Leadás}
	\def\filePath[#1]{./../../#1}
}{}

\makeatother

\chapter{Szkeleton tervezése}


\section{A szkeleton modell valóságos use-case-ei}

\subsection{Use-case diagram}
\begin{figure}[H] 
    \centering 
    \includegraphics[width=1\textwidth]{docs/Skeleton/diagrams/skeleton-use-case.eps} 
    \caption{} 
\end{figure} 

\subsection{Use-case leírások}
\begin{use-case}
    {Use-case Neve}
    {Az eset rövid leírása}
    {Aktorok} 
    \textbf{A.1} Alternatíva
\end{use-case}

\section{A szkeleton kezelői felületének terve, dialógusok}
\comment{A szkeleton által elfogadott bemenetek , valamint a szöveges konzolon megjelenő kimenetek. A kiemenet formátuma olyan kell legyen, ami alapján a működés összevethető a korábbi szekvencia-diagramokkal.}

\comment{Konzol input/output szemléltetésre:}
\begin{verbatim}
    > move
    get item
\end{verbatim}

\section{Szekvencia diagramok a belső működésre}
%\comment{A szkeletonban implementált szekvenciadiagramok. Tipikusan egy use-case egy diagram. Ezek megegyezhetnek a korábban specifikált diagramokkal, de az egyes életvonalakat (lifeline) egyértelműen a szkeletonban példányosított objektumokhoz kell tudni kötni. Azt kell megjeleníteni, hogy a szkeletonban létrehozott objektumok egymással hogyan fognak kommunikálni.}


\subsection{Controll Robots}

\begin{figure}[H] 
    \centering 
    \includegraphics[width=0.7\textwidth]{docs/Skeleton/diagrams/Controll-Robots/Move-Robots/Move-Robot-to-Asteroid.eps} 
    \caption{} 
\end{figure} 

\begin{figure}[H] 
    \centering 
    \includegraphics[width=0.7\textwidth]{docs/Skeleton/diagrams/Controll-Robots/Move-Robots/Move-Robot-Through-TG.eps} 
    \caption{} 
\end{figure} 


\begin{figure}[H] 
    \centering 
    \includegraphics[width=0.7\textwidth]{docs/Skeleton/diagrams/Controll-Robots/Robot-Drill/Robot-Drills-Asteroid-Succes.eps} 
    \caption{} 
\end{figure} 

\begin{figure}[H] 
    \centering 
    \includegraphics[width=0.7\textwidth]{docs/Skeleton/diagrams/Controll-Robots/Robot-Drill/Robot-Drills-Asteroid-Exploded.eps} 
    \caption{} 
\end{figure} 

\subsection{Craft Robot}

\begin{figure}[H] 
    \centering 
    \includegraphics[width=0.7\textwidth]{docs/Skeleton/diagrams/Craft_robot/sequence/1.CanCraft_robot.eps} 
    \caption{} 
\end{figure} 

\begin{figure}[H] 
    \centering 
    \includegraphics[width=0.7\textwidth]{docs/Skeleton/diagrams/Craft_robot/sequence/2.Possible_to_craft_robot.eps} 
    \caption{} 
\end{figure} 

\begin{figure}[H] 
    \centering 
    \includegraphics[width=0.7\textwidth]{docs/Skeleton/diagrams/Craft_robot/sequence/3.Craft_Robot.eps} 
    \caption{} 
\end{figure} 

\begin{figure}[H] 
    \centering 
    \includegraphics[width=0.7\textwidth]{docs/Skeleton/diagrams/Craft_robot/sequence/4.Cant_craft_robot.eps} 
    \caption{} 
\end{figure} 

\begin{figure}[H] 
    \centering 
    \includegraphics[width=0.7\textwidth]{docs/Skeleton/diagrams/Craft_robot/sequence/5.Not_possible_to_craft_robot.eps} 
    \caption{} 
\end{figure} 


\subsection{Drill}

\begin{figure}[H] 
    \centering 
    \includegraphics[width=0.7\textwidth]{docs/Skeleton/diagrams/Drill/SpaceShip-drill-succeed.eps} 
    \caption{} 
\end{figure} 

\begin{figure}[H] 
    \centering 
    \includegraphics[width=0.7\textwidth]{docs/Skeleton/diagrams/Drill/SpaceShip-drill-when-asteroid-exploded.eps} 
    \caption{} 
\end{figure} 

\begin{figure}[H] 
    \centering 
    \includegraphics[width=0.7\textwidth]{docs/Skeleton/diagrams/Drill/SpaceShip-drill-when-core-is-visible.eps} 
    \caption{} 
\end{figure} 

\subsection{Solar Flare}

\begin{figure}[H] 
    \centering 
    \includegraphics[width=0.75\textwidth]{docs/Skeleton/diagrams/Solar-Flare/Round-Ends-Robot.eps} 
    \caption{Itt látható hogy mi történik egy robottal, ha a Teszter befejezi a kört. Robot eset} 
\end{figure}

\begin{figure}[H] 
    \centering 
    \includegraphics[width=0.75\textwidth]{docs/Skeleton/diagrams/Solar-Flare/Round-Ends-SpaceShip.eps} 
    \caption{Itt látható hogy mi történik egy űrhajóval, ha a Teszter befejezi a kört.} 
\end{figure}


\subsection{Move Asteroid Field}

\begin{figure}[H] 
    \centering 
    \includegraphics[width=0.5\textwidth]{docs/Skeleton/diagrams/Move-Asteroid-Field/Change-Asteroid-Field-Distance.eps} 
    \caption{A tesztelő megváltoztatja az aszteroidamező távolságát} 
\end{figure}



\subsection{Place Gate}

\begin{figure}[H] 
    \centering 
    \includegraphics[width=1\textwidth]{docs/Skeleton/diagrams/Place-Gate/Place-first-gate.eps} 
    \caption{} 
\end{figure} 

\begin{figure}[H] 
    \centering 
    \includegraphics[width=1\textwidth]{docs/Skeleton/diagrams/Place-Gate/Place-second-gate.eps} 
    \caption{} 
\end{figure} 



\subsection{Place Resource}

\begin{figure}[H] 
    \centering 
    \includegraphics[width=1\textwidth]{docs/Skeleton/diagrams/Place-Resource/Asteroid-Exploded/Place-Resource-Asteroid-Exploded.eps} 
    \caption{} 
\end{figure} 

\begin{figure}[H] 
    \centering 
    \includegraphics[width=1\textwidth]{docs/Skeleton/diagrams/Place-Resource/Not-Drilled/PlaceResource-NotDrilled.eps} 
    \caption{} 
\end{figure} 

\section{Kommunikációs diagramok}
%\comment{A szkeletonban, az egyes szkeleton-use-case-ek futása során létrehozott objektumok és kapcsolataik bemutatására szolgáló diagramok. Ezek alapján valósítják meg a szkeleton fejlesztői az inicializáló kódrészleteket.}


\section{Napló}

\begin{naplo}

	\naplotag{2021.03.10. 10:00 }{ 2 óra }{ Csapat }
	{ 
		Konzultáció és előző heti munkánk értékelése, feladatok kiosztása   
        \newline
		Részletesen: \ref{appendix:meeting9}
	}

    
\end{naplo}

\begin{toappendix}

	\markdownInput[shiftHeadings=2]{\filePath[docs/meetings/task-breakdown-3.md]}

    \markdownInput[shiftHeadings=2]{\filePath[docs/meetings/meeting-9.md]}
	
\end{toappendix}

\end{document}