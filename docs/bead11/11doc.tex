\documentclass[../../projlab]{subfiles}
\begin{document}

\makeatletter

\ifSubfilesClassLoaded{
	\coverpage{10. Leadás}
	%\renewcommand{\filePath}[1]{./../../ #1}
	\def\filePath[#1]{./../../#1}
}{}

\makeatother

\chapter{Grafikus felület specifikációja}

\section{A grafikus interfész}
%\comment{A menürendszer, a kezelői felület grafikus képe. A grafikus felület megjelenését, a használt ikonokat, stb screenshot-szerű képekkel kell bemutatni. Az építészetben ez a homlokzati terv.}
\diagram{docs/bead11/img/InGame.png}{Példa kép}{15cm}

\section{A grafikus rendszer architektúrája}
\comment{A felület működésének elve, a grafikus rendszer architektúrája (struktúra diagramok). A struktúra diagramokon a prototípus azon és csak azon osztályainak is szerepelnie kell, amelyekhez a grafikus felületet létrehozó osztályok kapcsolódnak.}

\subsection{A felület működési elve}
\comment{Le kell írni, hogy a grafikai megjelenésért felelős osztályok, objektumok hogyan kapcsolódnak a meglevő rendszerhez, a megjelenítés során mi volt az alapelv. Törekedni kell az MVC megvalósításra. Alapelvek lehetnek: \textbf{push} alapú: a modell értesíti a felületet, hogy változott; \textbf{pull} alapú: a felület kérdezi le a modellt, hogy változott-e; \textbf{kevert}: a kettő kombinációja.}

\subsection{A felület osztály-struktúrája}
\begin{figure}[H] 
    \centering 
    \includegraphics[width=1\textwidth]{docs/bead11/class_diag} 
    \caption{Class diagram} 
\end{figure} 

\section{A grafikus objektumok felsorolása}
\comment{Az új osztályok felsorolása. Az régi osztályok közül azoknak a felsorolása, ahol változás volt. Ezek esetén csak a változásokat kell leírni.}

\subsection{Új osztályok}
\subsubsection{Osztály1}
\begin{class-template-responsibility}
    Felelősség leírása
\end{class-template-responsibility}
\begin{class-template-interface}
    Megvalósított interfészek felsorolása
\end{class-template-interface}
\begin{class-template-baseclass}
    Ős-Ősosztály \baseclass Ősosztály... 
\end{class-template-baseclass}
\begin{class-template-attribute}
    \classitem{+A [0..*]}{adattag A}
\end{class-template-attribute}
\comment{Milyen publikus, protected és privát  metódusokkal rendelkezik. Metódusonként precíz leírás, ha szükséges, activity diagram is a metódusban megvalósítandó algoritmusról. Minden olyan metódusnak szerepelnie kell, amelyiket az osztály megvalósít vagy felüldefiniál.}
\begin{class-template-method}
    \classitem{+B(A a) : void}{metódus B}
\end{class-template-method}

\subsection{Megváltozott osztályok}
\subsubsection{Osztály2}
\begin{class-template-responsibility}
    Felelősség leírása
\end{class-template-responsibility}
\begin{class-template-interface}
    Megvalósított interfészek felsorolása
\end{class-template-interface}
\begin{class-template-baseclass}
    Ős-Ősosztály \baseclass Ősosztály... 
\end{class-template-baseclass}
\begin{class-template-attribute}
    \classitem{+A [0..*]}{adattag A}
\end{class-template-attribute}
\comment{Milyen publikus, protected és privát  metódusokkal rendelkezik. Metódusonként precíz leírás, ha szükséges, activity diagram is a metódusban megvalósítandó algoritmusról. Minden olyan metódusnak szerepelnie kell, amelyiket az osztály megvalósít vagy felüldefiniál.}
\begin{class-template-method}
    \classitem{+B(A a) : void}{metódus B}
\end{class-template-method}

\section{Kapcsolat az alkalmazói rendszerrel}
\comment{Szekvencia-diagramokon ábrázolni kell a grafikus rendszer működését. Konzisztens kell legyen az előző alfejezetekkel. Minden metódus, ami ott szerepel, fel kell tűnjön valamelyik szekvenciában. Minden metódusnak, ami szekvenciában szerepel, szereplnie kell a valamelyik osztálydiagramon.}
\begin{figure}[H] 
    \centering 
    \includegraphics[width=1\textwidth]{docs/bead11/seq/addlistener} 
    \caption{Class diagram} 
\end{figure} 

\begin{figure}[H] 
    \centering 
    \includegraphics[width=1\textwidth]{docs/bead11/seq/notifylisteners} 
    \caption{Class diagram} 
\end{figure} 

\begin{figure}[H] 
    \centering 
    \includegraphics[width=1\textwidth]{docs/bead11/seq/notifylistenersEngine} 
    \caption{Class diagram} 
\end{figure} 

\begin{figure}[H] 
    \centering 
    \includegraphics[width=1\textwidth]{docs/bead11/seq/notifylistenersManager} 
    \caption{Class diagram} 
\end{figure} 

\begin{figure}[H] 
    \centering 
    \includegraphics[width=1\textwidth]{docs/bead11/seq/paint} 
    \caption{Class diagram} 
\end{figure} 

\begin{figure}[H] 
    \centering 
    \includegraphics[width=1\textwidth]{docs/bead11/seq/removelistener} 
    \caption{Class diagram} 
\end{figure}

\begin{figure}[H] 
    \centering 
    \includegraphics[width=1\textwidth]{docs/bead11/seq/selectionChanged} 
    \caption{Class diagram} 
\end{figure}

\begin{figure}[H] 
    \centering 
    \includegraphics[width=1\textwidth]{docs/bead11/seq/stateChanged} 
    \caption{Class diagram} 
\end{figure}

\begin{figure}[H] 
    \centering 
    \includegraphics[width=0.5\textwidth]{docs/bead11/seq/stateChanged2} 
    \caption{Class diagram} 
\end{figure}

\begin{figure}[H] 
    \centering 
    \includegraphics[width=1\textwidth]{docs/bead11/seq/stateChanged3} 
    \caption{Class diagram} 
\end{figure}


\newpage
\section{Napló}

\begin{naplo}
    \naplotag{feb. 18. 16h }{ 1 óra }{ Csapat }{ Értekezlet  \newline Döntés: Segédeszközök kiválasztása (git, trello, drive)}
\end{naplo}


\end{document}