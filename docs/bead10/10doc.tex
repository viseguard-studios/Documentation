\documentclass[../../projlab]{subfiles}
\begin{document}

\makeatletter

\ifSubfilesClassLoaded{
	\coverpage{10. Leadás}
	%\renewcommand{\filePath}[1]{./../../ #1}
	\def\filePath[#1]{./../../#1}
}{}

\makeatother


\chapter{Prototípus beadása}

\section{Fordítási és futtatási útmutató}
\comment{A feltöltött program fordításával és futtatásával kapcsolatos útmutatás. Ennek tartalmaznia kell leltárszerűen az egyes fájlok pontos nevét, méretét byte-ban, keletkezési idejét, valamint azt, hogy a fájlban mi került megvalósításra.}

\subsection{Fájllista}

\begin{fajllista}
    \fajl
    {Demo.java} % Kezdet
    {353 byte} % Méret
    {2020.03.26~21:05~} % Időpont
    {Demo programosztály} % Leírás
\end{fajllista}

\subsection{Fordítás}
\comment{A fenti listában szereplő forrásfájlokból milyen műveletekkel lehet a bináris, futtatható kódot előállítani. Az előállításhoz csak a 2. Követelmények c. dokumentumban leírt környezetet szabad előírni.}

\begin{verbatim}
javac -d bin *.java
\end{verbatim}

\subsection{Futtatás}
\comment{A futtatható kód elindításával kapcsolatos teendők leírása. Az indításhoz csak a 2. Követelmények c. dokumentumban leírt környezetet szabad előírni.}

\begin{verbatim}
cd bin
java Main.java
\end{verbatim}

\section{Tesztek jegyzőkönyvei}

\subsection{Teszteset1}
\comment{Az alábbi táblázatot a megismételt (hibás) tesztek esetén kell kitölteni minden ismétléshez egyszer. Ha szükséges, akkor a valós kimenet is mellékelhető mint a teszt eredménye}
\testFAIL{????}{Ápr. 25. 13:50}{ A teszt eredménye}{A teszt hibás futásának okai}{Változtatások}
\comment{Az alábbi táblázatot az utolsó, sikeres tesztfuttatáshoz kell kitölteni}
\testOK{????}{Ápr. 25. 14:00}

\section{Értékelés}
\comment{A projekt kezdete óta az értékelésig eltelt időben tagokra bontva, százalékban.}
\begin{ertekeles}
    \ertekelestag{Teszt János}{ ?????? }{100\%}
\end{ertekeles}


\section{Napló}

\begin{naplo}
    \naplotag{feb. 18. 16h }{ 1 óra }{ Csapat }{ Értekezlet  \newline Döntés: Segédeszközök kiválasztása (git, trello, drive)}
\end{naplo}

\end{document}