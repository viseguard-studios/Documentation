\documentclass[../../projlab]{subfiles}
\begin{document}

\makeatletter

\ifSubfilesClassLoaded{
	\coverpage{10. Leadás}
	%\renewcommand{\filePath}[1]{./../../ #1}
	\def\filePath[#1]{./../../#1}
}{}

\makeatother


\chapter{Prototípus beadása}

\section{Fordítási és futtatási útmutató}
%\comment{A feltöltött program fordításával és futtatásával kapcsolatos útmutatás. Ennek tartalmaznia kell leltárszerűen az egyes fájlok pontos nevét, méretét byte-ban, keletkezési idejét, valamint azt, hogy a fájlban mi került megvalósításra.}

\subsection{Fájllista}

\begin{fajllista}
    \fajl
    {Building.java} % Kezdet
    {1 KB} % Méret
    {2021.04.21~12:42~} % Időpont
    {Épületek programosztálya} % Leírás
	\fajl
    {SpaceStation.java} % Kezdet
    {2 KB} % Méret
    {2021.04.21~12:42~} % Időpont
    {Űrállomás programosztálya} % Leírás
	\fajl
    {TeleportGate.java} % Kezdet
    {6 KB} % Méret
    {2021.04.21~12:42~} % Időpont
    {Teleportkapuk programosztálya} % Leírás
	\fajl
    {Robot.java} % Kezdet
    {2 KB} % Méret
    {2021.04.21~12:42~} % Időpont
    {Robotok programosztálya} % Leírás
	\fajl
    {SpaceShip.java} % Kezdet
    {4 KB} % Méret
    {2021.04.21~12:42~} % Időpont
    {Telepesek programosztálya} % Leírás
	\fajl
    {UFO.java} % Kezdet
    {2 KB} % Méret
    {2021.04.21~12:42~} % Időpont
    {UFOk programosztálya} % Leírás
	\fajl
    {Vessel.java} % Kezdet
    {5 KB} % Méret
    {2021.04.21~12:42~} % Időpont
    {Űrjárművek általános programosztálya} % Leírás
	\fajl
    {Asteroid.java} % Kezdet
    {14 KB} % Méret
    {2021.04.21~12:42~} % Időpont
    {Aszteroidák programosztálya} % Leírás
	\fajl
    {Entity.java} % Kezdet
    {2 KB} % Méret
    {2021.04.21~12:42~} % Időpont
    {Entitások általános programosztálya} % Leírás
	\fajl
    {MovableEntity.java} % Kezdet
    {2 KB} % Méret
    {2021.04.21~12:42~} % Időpont
    {Mozgásra képes entitások általános programosztálya} % Leírás
	\fajl
    {AsteroidInventory.java} % Kezdet
    {3 KB} % Méret
    {2021.04.21~12:42~} % Időpont
    {Aszteroidák raktárának programosztálya} % Leírás
	\fajl
    {IInventory.java} % Kezdet
    {2 KB} % Méret
    {2021.04.21~12:42~} % Időpont
    {Raktár interfész} % Leírás
	\fajl
    {InfiniteInventory.java} % Kezdet
    {2 KB} % Méret
    {2021.04.21~12:42~} % Időpont
    {UFOk raktárának programosztálya} % Leírás
	\fajl
    {SSInventory.java} % Kezdet
    {4 KB} % Méret
    {2021.04.21~12:42~} % Időpont
    {Telepesek raktárának programosztálya} % Leírás
	\fajl
    {Coal.java} % Kezdet
    {1 KB} % Méret
    {2021.04.21~12:42~} % Időpont
    {Szén programosztálya} % Leírás
	\fajl
    {Ice.java} % Kezdet
    {2 KB} % Méret
    {2021.04.21~12:42~} % Időpont
    {Vízjég programosztálya} % Leírás
	\fajl
    {Iron.java} % Kezdet
    {1 KB} % Méret
    {2021.04.21~12:42~} % Időpont
    {Vas programosztálya} % Leírás
	\fajl
    {Resource.java} % Kezdet
    {1 KB} % Méret
    {2021.04.21~12:42~} % Időpont
    {Nyersanyagok általános programosztálya} % Leírás
	\fajl
    {Titan.java} % Kezdet
    {1 KB} % Méret
    {2021.04.21~12:42~} % Időpont
    {Titánium programosztálya} % Leírás
	\fajl
    {Uranium.java} % Kezdet
    {2 KB} % Méret
    {2021.04.21~12:42~} % Időpont
    {Uránium programosztálya} % Leírás
	\fajl
    {Item.java} % Kezdet
    {2 KB} % Méret
    {2021.04.21~12:42~} % Időpont
    {Raktárelemek programosztálya} % Leírás
	\fajl
    {TeleportGateItem.java} % Kezdet
    {2 KB} % Méret
    {2021.04.21~12:42~} % Időpont
    {Teleportkapu-elemek programosztálya} % Leírás
	\fajl
    {Recipe.java} % Kezdet
    {5 KB} % Méret
    {2021.04.21~12:42~} % Időpont
    {Receptek általános programosztálya} % Leírás
	\fajl
    {RobotRecipe.java} % Kezdet
    {2 KB} % Méret
    {2021.04.21~12:42~} % Időpont
    {Robot receptjének programosztálya} % Leírás
	\fajl
    {SpaceStationRecipe.java} % Kezdet
    {2 KB} % Méret
    {2021.04.21~12:42~} % Időpont
    {Űrállomás receptjének programosztálya} % Leírás
	\fajl
    {TeleportGateRecipe.java} % Kezdet
    {3 KB} % Méret
    {2021.04.21~12:42~} % Időpont
    {Teleportkapuk receptjének programosztálya} % Leírás
	\fajl
    {Engine.java} % Kezdet
    {3 KB} % Méret
    {2021.04.21~12:42~} % Időpont
    {Játékmotor programosztálya} % Leírás
	\fajl
    {GameManager.java} % Kezdet
    {9 KB} % Méret
    {2021.04.21~12:42~} % Időpont
    {Játékmenedzser programosztálya} % Leírás
	\fajl
    {Main.java} % Kezdet
    {1 KB} % Méret
    {2021.04.21~12:42~} % Időpont
    {Főosztály} % Leírás
	\fajl
    {Player.java} % Kezdet
    {2 KB} % Méret
    {2021.04.21~12:42~} % Időpont
    {Játékosok programosztálya} % Leírás
	\fajl
    {Scene.java} % Kezdet
    {3 KB} % Méret
    {2021.04.21~12:42~} % Időpont
    {Játéktér programosztálya} % Leírás
	\fajl
    {Command.java} % Kezdet
    {1 KB} % Méret
    {2021.04.21~12:42~} % Időpont
    {Általános commandok programosztálya} % Leírás
	\fajl
    {CommandExecutor.java} % Kezdet
    {2 KB} % Méret
    {2021.04.21~12:42~} % Időpont
    {Commandok rendszerezésének, végrehajtásának programosztálya} % Leírás
	\fajl
    {CraftCmd.java} % Kezdet
    {3 KB} % Méret
    {2021.04.21~12:42~} % Időpont
    {Készítés commandjának programosztálya} % Leírás
	\fajl
    {DrillCmd.java} % Kezdet
    {2 KB} % Méret
    {2021.04.21~12:42~} % Időpont
    {Fúrás commandjának programosztálya} % Leírás
	\fajl
    {EndturnCmd.java} % Kezdet
    {1 KB} % Méret
    {2021.04.21~12:42~} % Időpont
    {Kör vége commandjának programosztálya} % Leírás
	\fajl
    {HelpCmd.java} % Kezdet
    {1 KB} % Méret
    {2021.04.21~12:42~} % Időpont
    {Commandok listázásának programosztálya} % Leírás
	\fajl
    {HideCmd.java} % Kezdet
    {2 KB} % Méret
    {2021.04.21~12:42~} % Időpont
    {Elbújás commandjának programosztálya} % Leírás
	\fajl
    {MineCmd.java} % Kezdet
    {2 KB} % Méret
    {2021.04.21~12:42~} % Időpont
    {Bányászat commandjának programosztálya} % Leírás
	\fajl
    {MoveCmd.java} % Kezdet
    {2 KB} % Méret
    {2021.04.21~12:42~} % Időpont
    {Mozgás commandjának programosztálya} % Leírás
	\fajl
    {NeighboursCmd.java} % Kezdet
    {2 KB} % Méret
    {2021.04.21~12:42~} % Időpont
    {Szomszédságok commandjának programosztálya} % Leírás
	\fajl
    {PickupCmd.java} % Kezdet
    {2 KB} % Méret
    {2021.04.21~12:42~} % Időpont
    {Item felszedése commandjának programosztálya} % Leírás
	\fajl
    {PlaceCmd.java} % Kezdet
    {2 KB} % Méret
    {2021.04.21~12:42~} % Időpont
    {Item visszahelyezése commandjának programosztálya} % Leírás
	\fajl
    {SolarDistCmd.java} % Kezdet
    {1 KB} % Méret
    {2021.04.21~12:42~} % Időpont
    {Naptávolság commandjának programosztálya} % Leírás
	\fajl
    {SolarFlareCmd.java} % Kezdet
    {1 KB} % Méret
    {2021.04.21~12:42~} % Időpont
    {Napvihar commandjának programosztálya} % Leírás
	\fajl
    {StartgameCmd.java} % Kezdet
    {2 KB} % Méret
    {2021.04.21~12:42~} % Időpont
    {Játékindítás commandjának programosztálya} % Leírás
	\fajl
    {StatusCmd.java} % Kezdet
    {2 KB} % Méret
    {2021.04.21~12:42~} % Időpont
    {Státuszlekérdezés commandjának programosztálya} % Leírás
	\fajl
    {Namer.java} % Kezdet
    {1 KB} % Méret
    {2021.04.21~12:42~} % Időpont
    {Névgenerálás programosztálya} % Leírás
	\fajl
	{RandomCollection.java} % Kezdet
    {1 KB} % Méret
    {2021.04.21~12:42~} % Időpont
    {Súlyozott lista programosztálya} % Leírás
	\fajl
	{Vector2.java} % Kezdet
    {1 KB} % Méret
    {2021.04.21~12:42~} % Időpont
    {Pozíció típusának programosztálya} % Leírás
	\fajl
	{StationCreator.java} % Kezdet
    {1 KB} % Méret
    {2021.04.21~12:42~} % Időpont
    {Űrállomás létrehozás programosztálya} % Leírás
	\fajl
	{TeleGateCreator.java} % Kezdet
    {2 KB} % Méret
    {2021.04.21~12:42~} % Időpont
    {Teleportkapu létrehozás programosztálya} % Leírás
	\fajl
	{RobotCreator.java} % Kezdet
    {3 KB} % Méret
    {2021.04.21~12:42~} % Időpont
    {Robot létrehozás programosztálya} % Leírás
	\fajl
	{SpaceShipCreator.java} % Kezdet
    {4 KB} % Méret
    {2021.04.21~12:42~} % Időpont
    {Telepes létrehozás programosztálya} % Leírás
	\fajl
	{UfoCreator.java} % Kezdet
    {3 KB} % Méret
    {2021.04.21~12:42~} % Időpont
    {Ufo létrehozás programosztálya} % Leírás
	\fajl
	{AsteroidCreator.java} % Kezdet
    {6 KB} % Méret
    {2021.04.21~12:42~} % Időpont
    {Aszteroida létrehozás programosztálya} % Leírás
	\fajl
	{BuildingCreator.java} % Kezdet
    {3 KB} % Méret
    {2021.04.21~12:42~} % Időpont
    {Épület létrehozás programosztálya} % Leírás
	\fajl
	{MovableEntityCreator.java} % Kezdet
    {2 KB} % Méret
    {2021.04.21~12:42~} % Időpont
    {Mozgásra képes entitás létrehozás programosztálya} % Leírás
	\fajl
	{VesselCreator.java} % Kezdet
    {4 KB} % Méret
    {2021.04.21~12:42~} % Időpont
    {Űrjármű létrehozás programosztálya} % Leírás
	\fajl
	{CoalCreator.java} % Kezdet
    {1 KB} % Méret
    {2021.04.21~12:42~} % Időpont
    {Szén létrehozás programosztálya} % Leírás
	\fajl
	{IceCreator.java} % Kezdet
    {1 KB} % Méret
    {2021.04.21~12:42~} % Időpont
    {Vízjég létrehozás programosztálya} % Leírás
	\fajl
	{IronCreator.java} % Kezdet
    {1 KB} % Méret
    {2021.04.21~12:42~} % Időpont
    {Vas létrehozás programosztálya} % Leírás
	\fajl
	{TeleItemCreator.java} % Kezdet
    {1 KB} % Méret
    {2021.04.21~12:42~} % Időpont
    {Teleportkapu item létrehozás programosztálya} % Leírás
	\fajl
	{TitaniumCreator.java} % Kezdet
    {1 KB} % Méret
    {2021.04.21~12:42~} % Időpont
    {Titán létrehozás programosztálya} % Leírás
	\fajl
	{UraniumCreator.java} % Kezdet
    {1 KB} % Méret
    {2021.04.21~12:42~} % Időpont
    {Urán létrehozás programosztálya} % Leírás
	\fajl
	{EntityCreator.java} % Kezdet
    {1 KB} % Méret
    {2021.04.21~12:42~} % Időpont
    {Entitás létrehozás programosztálya} % Leírás
	\fajl
	{FileOpener.java} % Kezdet
    {8 KB} % Méret
    {2021.04.21~12:42~} % Időpont
    {Fájl megnyitás programosztálya} % Leírás
	\fajl
	{GMCreator.java} % Kezdet
    {6 KB} % Méret
    {2021.04.21~12:42~} % Időpont
    {GameManager létrehozás programosztálya} % Leírás
	\fajl
	{ItemCreator.java} % Kezdet
    {3 KB} % Méret
    {2021.04.21~12:42~} % Időpont
    {Item létrehozás programosztálya} % Leírás
	\fajl
	{MrConnector.java} % Kezdet
    {2 KB} % Méret
    {2021.04.21~12:42~} % Időpont
    {Aszteroidakapcsolatok építésének programosztálya} % Leírás
	\fajl
	{PlayerCreator.java} % Kezdet
    {2 KB} % Méret
    {2021.04.21~12:42~} % Időpont
    {Játékos létrehozás programosztálya} % Leírás
	\fajl
	{SceneCreator.java} % Kezdet
    {1 KB} % Méret
    {2021.04.21~12:42~} % Időpont
    {Játéktér létrehozás programosztálya} % Leírás
\end{fajllista}

\subsection{Fordítás}
%\comment{A fenti listában szereplő forrásfájlokból milyen műveletekkel lehet a bináris, futtatható kódot előállítani. Az előállításhoz csak a 2. Követelmények c. dokumentumban leírt környezetet szabad előírni.}
A fenti fájlokból a javac parancs kiadásával fordítható a tesztelőprogram (az alábbi módon).
Esetleg GUI-val rendelkező fejlesztőkörnyezetből, az annak megfelelő módon.

\begin{verbatim}
    javac -d bin *.java
\end{verbatim}

\subsection{Futtatás}
%\comment{A futtatható kód elindításával kapcsolatos teendők leírása. Az indításhoz csak a 2. Követelmények c. dokumentumban leírt környezetet szabad előírni.}

Futtatáshoz a következő parancsokat kell kiadni.
\begin{verbatim}
    cd bin
    java SkeletonEntry
\end{verbatim}

\section{Tesztek jegyzőkönyvei}

\subsection{Complex test}
\begin{test-case-description}
    Ez a teszt a Teamses labor csatornában lévő példa alapján készült el.
\end{test-case-description}
\begin{test-case-function}
    Egy összetetteb teszteset vizsgálata
\end{test-case-function}
\begin{test-case-input}
A pálya betöltése:
    \begin{verbatim}
startgame 0 komplex_test.vsgstd 1
mine s1
endturn
move s1
1
endturn
place from s1
1
solarDistance 1
endturn
    \end{verbatim}


\end{test-case-input}
\begin{test-case-output}
\begin{verbatim}
Starting game
Deterministic game

a1 mined 
* Item: Uranium Exp:2

Neighbours:
0 - a2
1 - a3

Items:
1. Uranium Exp:2

Placed Uranium Exp:2

Solar distance changed to 1

[EVENT] a3 exploded
[EVENT] s1 exploded
[EVENT] g2 exploded
\end{verbatim}
\end{test-case-output}

\subsection{Drilling and Mining}
\begin{test-case-description}
    A bányászás mozzanatainak tesztelésére.
\end{test-case-description}
\begin{test-case-function}
    Egy aszteroidán tartózkodó SpaceShip megfúrja azt és kibányássza a nyersanyagot. \newline
    Működés nagyban függ a felhasználó által választott értékeken, így bizonyos lépések ismétlése szükséges a várt eredmény eléréséhez.
\end{test-case-function}
\begin{test-case-input}
\begin{verbatim}
startgame 0 test_files/mining_test.vsgstd 1
drill s1
endturn
mine s1
endturn
\end{verbatim}
\end{test-case-input}
\begin{test-case-output}

\begin{verbatim}
Created Engine
Application started

Starting game
Deterministic game

a1 crust = 0

a1 mined 
* Item: Iron
\end{verbatim}
\end{test-case-output}
\testOK {Csapat}{Ápr. 21. 13:00}



\subsection{Craft Win condition}
\begin{test-case-description}
    A tárgyak előállításának tesztelésére.
\end{test-case-description}
\begin{test-case-function}
    Egy aszteroidán tartózkodó SpaceShip megépít egy Űrállomást. \newline
    Működés nagyban függ a felhasználó által választott értékeken, így bizonyos lépések ismétlése szükséges a várt eredmény eléréséhez.
\end{test-case-function}
\begin{test-case-input}
   
    \begin{verbatim}
    startgame 0 crafting_test.vsgstd 1
craft s1 Base
endturn
    \end{verbatim}
\end{test-case-input}
\begin{test-case-output}
\begin{verbatim}
    
    Created Engine
Application started

Starting game
Deterministic game

Current game is over
You successfully crafted a Base
Congratulations, you won the game!

    
\end{verbatim}
\end{test-case-output}
\testOK {Csapat}{Ápr. 21. 13:00}

\subsection{Move and Hide}
\begin{test-case-description}
    A tárgyak előállításának tesztelésére.
\end{test-case-description}
\begin{test-case-function}
    Egy aszteroidán tartózkodó SpaceShip egy kapun keresztül ellátogat egy másik aszteroidára és ott elbújik. \newline
    Működés nagyban függ a felhasználó által választott értékeken, így bizonyos lépések ismétlése szükséges a várt eredmény eléréséhez.
\end{test-case-function}
\begin{test-case-input}

    
    \begin{verbatim}
    
    \end{verbatim}
\end{test-case-input}
\begin{test-case-output}

\end{test-case-output}
\testOK {Csapat}{Ápr. 21. 13:00}

\subsection{UFO/Robot Moving and Drilling/Mining}
\begin{test-case-description}
    A jáékosok által NEM vezérelt repülő tárgyak képességeinek ellenőrzése.
\end{test-case-description}
\begin{test-case-function}
    Egy aszteroidán tartózkodó UFO és Robot egy kapun keresztül ellátogat egy másik aszteroidára és ott a robot fúrni az UFO pedig bányászni kezd. \newline
    Működés nagyban függ a felhasználó által választott értékeken, így bizonyos lépések ismétlése szükséges a várt eredmény eléréséhez.
\end{test-case-function}
\begin{test-case-input}

    A világ betöltése: 
    \begin{verbatim}
    >startgame 0 UFO_test.vsgstd
    \end{verbatim}
    U1 megnézi a szomszédokat:
    \begin{verbatim}
    >neighbors U1
    Neighbours:
    01 - Asteroid_1  - ?
    \end{verbatim}
    U1 és R1 kiválasztja a mozgás végpontját:
    \begin{verbatim}
    >move U1
    01 - Asteroid_1
    >1
    >move R1
    01 - Asteroid_1
    >1
    \end{verbatim}
    R1 fúrni kezd:
    \begin{verbatim}
    >drill R1
    A1 crust = 0
    >endturn
    Next Player: player_1
    \end{verbatim}
    U1 bányászni kezd:
    \begin{verbatim}
    >mine U1
    A1 mined
    * Item: Iron
    >endturn
    Next Player: player_1
    \end{verbatim}
\end{test-case-input}
\begin{test-case-output}

\end{test-case-output}
\testOK {Csapat}{Ápr. 21. 13:00}


\section{Értékelés}
%\comment{A projekt kezdete óta az értékelésig eltelt időben tagokra bontva, százalékban.}
\begin{ertekeles}
    \ertekelestag{Sike Ádám}{ E8Z277 }{18\%}
    \ertekelestag{Dömötör Péter}{ G2Y5TI }{28\%}
    \ertekelestag{Gao Tong}{ I2SVOS }{18\%}
    \ertekelestag{Nagy Beáta}{ GPOGC5 }{18\%}
    \ertekelestag{Tatai Titusz Miklós}{ IJHLYX }{18\%}
\end{ertekeles}


\section{Napló}

\begin{naplo}

    \naplotag{2021.04.08. 16:00 }{ 4 óra }{ Gao Tong }
	{ 
		Az előző leadásnál lemaradt - Osztályleírások
	}
    \naplotag{2021.04.09. 16:00 }{ 4 óra }{ Gao Tong}
	{ 
		Az előző leadásnál lemaradt - Osztályleírások
	}
    \naplotag{2021.04.14. 10:15 }{ 10 perc }{ Csapat }
	{ 
		Rövid meeting 
	}
     \naplotag{2021.04.19. 21:00 }{ 2 óra }{ Dömötör Péter }
	{ 
		Refaktorozás
	}
    \naplotag{2021.04.16. 18:00 }{ 2 óra }{ Tatai Titusz }
	{ 
		Fájl beolvasás - alapok
	}
	\naplotag{2021.04.17. 14:15 }{ 10 perc }{ Csapat }
	{ 
		Rövid meeting 
	}
    \naplotag{2021.04.17. 14:00 }{ 6 óra }{ Nagy Beáta }
	{ 
		Receptek
	}
    \naplotag{2021.04.17. 12:00 }{ 4 óra }{ Sike Ádám }
	{ 
		Fájl beolvasás - szövegfeldolgozás
	}
     \naplotag{2021.04.17. 15:00 }{ 6 óra }{Dömötör Péter }
	{ 
		Parancsok létrehozása, compile, világgenerálás
	}
    \naplotag{2021.04.17. 16:00 }{ 4 óra }{ Tatai Titusz }
	{ 
		Ufók működése
	}
    \naplotag{2021.04.17. 17:00 }{ 3 óra }{ Gao Tong}
	{ 
		Apróbb munkálatok
	}
    \naplotag{2021.04.18. 18:00 }{ 2 óra }{ Sike Ádám }
	{ 
		Fájl beolvasás - fájl levelei
	}
     \naplotag{2021.04.18. 20:00 }{ 2 óra }{ Nagy Beáta }
	{ 
		Entitás implementáció
	}
    \naplotag{2021.04.19. 20:00 }{ 2 óra }{ Sike Ádám }
	{ 
		Fájl beolvasás - Scene létrehozása
	}
    \naplotag{2021.04.20. 10:00 }{ 4 óra }{ Csapat }
	{ 
		Befejező lépések
	}
	\naplotag{2021.04.20. 20:00 }{ 4 óra }{ Csapat }
	{ 
		Befejező lépések II.
	}
	\naplotag{2021.04.21. 8:00 }{ 2 óra }{ Csapat }
	{ 
		Befejező lépések III.
	}
	\naplotag{2021.04.21. 12:00 }{ 2 óra }{ Csapat }
	{ 
		Befejező lépések III-final
	}

\end{naplo}

\end{document}