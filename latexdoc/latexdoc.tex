\documentclass[]{article}
\usepackage[margin=1in]{geometry}
\usepackage[utf8]{inputenc}
\usepackage[magyar]{babel}
\usepackage[hidelinks]{hyperref} 
\usepackage{graphicx}
\usepackage{comment}
\usepackage{amsthm}
\usepackage{MnSymbol}
\usepackage{wasysym}
\usepackage{fancyhdr}
\usepackage{longtable}
\usepackage{changepage}
\usepackage{xcolor}
\usepackage{algorithmicx}
\usepackage{verbatim}
\usepackage{listings}
\definecolor{codegreen}{rgb}{0,0.6,0}
\definecolor{codegray}{rgb}{0.5,0.5,0.5}
\definecolor{codepurple}{rgb}{0.58,0,0.82}
\definecolor{backcolour}{rgb}{0.95,0.95,0.92}
\lstset{
    language = [LaTeX]{TeX},
    moretexcs    = {diagram, naplotag, subsection, test, ertekelestag, ertekelestagk, testOK, testFAIL, classitem, fajl, tervitem, szotaritem, baseclass },
    backgroundcolor=\color{backcolour},   
    keywordstyle=\color{codegreen},
    numberstyle=\tiny\color{codegray},
    stringstyle=\color{codepurple},
    basicstyle=\ttfamily\footnotesize,
    breakatwhitespace=false,         
    breaklines=true,                 
    captionpos=b,                    
    keepspaces=true,                 
    numbers=left,                    
    numbersep=5pt,                  
    showspaces=false,                
    showstringspaces=false,
    showtabs=false,                  
    tabsize=2,
    literate=%
        {á}{{\'a}}1 {é}{{\'e}}1 {í}{{\'i}}1 {ó}{{\'o}}1 {ú}{{\'u}}1
        {Á}{{\'A}}1 {É}{{\'E}}1 {Í}{{\'I}}1 {Ó}{{\'O}}1 {Ú}{{\'U}}1
        {à}{{\`a}}1 {è}{{\`e}}1 {ì}{{\`i}}1 {ò}{{\`o}}1 {ù}{{\`u}}1
        {À}{{\`A}}1 {È}{{\'E}}1 {Ì}{{\`I}}1 {Ò}{{\`O}}1 {Ù}{{\`U}}1
        {ä}{{\"a}}1 {ë}{{\"e}}1 {ï}{{\"i}}1 {ö}{{\"o}}1 {ü}{{\"u}}1
        {Ä}{{\"A}}1 {Ë}{{\"E}}1 {Ï}{{\"I}}1 {Ö}{{\"O}}1 {Ü}{{\"U}}1
        {â}{{\^a}}1 {ê}{{\^e}}1 {î}{{\^i}}1 {ô}{{\^o}}1 {û}{{\^u}}1
        {Â}{{\^A}}1 {Ê}{{\^E}}1 {Î}{{\^I}}1 {Ô}{{\^O}}1 {Û}{{\^U}}1
        {Ã}{{\~A}}1 {ã}{{\~a}}1 {Õ}{{\~O}}1 {õ}{{\~o}}1
        {œ}{{\oe}}1 {Œ}{{\OE}}1 {æ}{{\ae}}1 {Æ}{{\AE}}1 {ß}{{\ss}}1
        {ű}{{\H{u}}}1 {Ű}{{\H{U}}}1 {ő}{{\H{o}}}1 {Ő}{{\H{O}}}1
        {ç}{{\c c}}1 {Ç}{{\c C}}1 {ø}{{\o}}1 {å}{{\r a}}1 {Å}{{\r A}}1
        {€}{{\euro}}1 {£}{{\pounds}}1 {«}{{\guillemotleft}}1
        {»}{{\guillemotright}}1 {ñ}{{\~n}}1 {Ñ}{{\~N}}1 {¿}{{?`}}1
}
\newenvironment{metaverbatim}{\verbatim}{\endverbatim}

\title{Dokumentáció a projekt laborítórium \LaTeX$ $ sablonjaihoz. }

\begin{document}
\maketitle
%---------------------------------------------------------------
%\\\\\\\\\\\\\\\\\\\\\\\\NE MÓDOSÍTSD///////////////////////////

\newenvironment{naplo}{\begin{longtable}{|l|l|p{2.5cm}|p{7.5cm}|}
        \hline
        \hline \multicolumn{1}{|c|}{\textbf{Kezdet}} & \multicolumn{1}{c|}{\textbf{Időtartam}} & \multicolumn{1}{c|}{\textbf{Résztvevők}} & \multicolumn{1}{c|}{\textbf{Leírás}}  \\ \hline 
        \endfirsthead
        
        \multicolumn{3}{c}%
        {{\bfseries}} \\
        \hline \multicolumn{1}{|c|}{\textbf{Kezdet}} & \multicolumn{1}{c|}{\textbf{Időtartam}} & \multicolumn{1}{c|}{\textbf{Résztvevők}} & \multicolumn{1}{c|}{\textbf{Leírás}}  \\ \hline 
        \endhead
        
}{\end{longtable}}

\newcommand{\naplotag}[4]{#1 & #2 & #3 & #4 \\
    \hline}

\newtheorem{defi}{Definíció}
\newtheorem{rov}{Rövidítés}
\newcommand{\urlref}[2]{
	\parbox{\linewidth}{\href{#1}{\RIGHTarrow \quad #2 - \texttt{#1}}}}

\newcommand{\diagram}[3]{
	\begin{figure}[h]
		\centering
		\includegraphics[width=#3]{#1}
		\caption{#2}
	\end{figure}}

\renewcommand{\comment}[1]{\begin{flushright}
		\color{red} 
		#1
\end{flushright} }

\newenvironment{class-template}[5]
{ 
    \hspace*{0.4em} $\blacksquare $ \large Felelősség \\  \normalsize #1 \\[12px]
    \hspace*{0.4em} $\blacksquare $ \large Ősosztályok \\  \normalsize #2 \\[12px]
    \hspace*{0.4em} $\blacksquare $ \large Interfészek \\  \normalsize #3 \\[12px]
    \hspace*{0.4em} $\blacksquare $ \large Attribútumok \\  \normalsize #4 \\[12px]
    \hspace*{0.4em} $\blacksquare $ \large Metódusok \\  \normalsize #5 \\[12px]
}{}

\newenvironment{class-template-responsibility}{\hspace*{0.4em} $\blacksquare $ \large Felelősség \\[5px]  \normalsize}{\\[12px]}
\newenvironment{class-template-baseclass}{\hspace*{0.4em} $\blacksquare $ \large Ősosztály \\[5px]  \normalsize}{\\[12px]}
\newenvironment{class-template-interface}{\hspace*{0.4em} $\blacksquare $ \large Interfészek \\[5px]  \normalsize}{\\[12px]}
\newenvironment{class-template-method}{\hspace*{0.4em} $\blacksquare $ \large Metódusok \\[5px]  \normalsize}{\\[12px]}
\newenvironment{class-template-attribute}{\hspace*{0.4em} $\blacksquare $ \large Attribútumok \\[5px]  \normalsize}{\\[12px]}
\newenvironment{class-template-statechart}{\hspace*{0.4em} $\blacksquare$ \large Állapotgép \\[5px] \normalsize}{\\[12px]}

\newenvironment{enum-template}[2]
{ 
    \hspace*{0.4em} $\blacksquare $ \large Felelősség \\[5px]  \normalsize #1 \\[12px]
    \hspace*{0.4em} $\blacksquare $ \large Literálok \\[5px]  \normalsize #2 \\[12px]
}{}	

{\vspace*{-1cm}}
\newcommand{\classitem}[2]{
    \hspace*{25px}\parbox{0.9\linewidth}{$\Diamond $ \textit{#1} - #2}
}

\newcommand{\baseclass}{ $\rightarrow$ }

\newenvironment{funkovetelmeny}[6]
{
    \noindent\begin{tabular}{| p{2cm} | p{2cm} | p{2.2cm} | p{5.3cm} | p{2.5cm} |}
        \hline
        \hline
        \textbf{Azonosító} & \textbf{Prioritás} & \textbf{Forrás} & \textbf{Use-case} & \textbf{Ellenőrzés} \\
        \hline
        #1 & #2 & #3 & #4 & #5 \\
        \hline
    \end{tabular}
    \\[1ex]
    \textbf{Leírás:} #6 \\[0.2ex]
    \itshape}
{\\[5px]}

\newenvironment{kovetelmeny}[5]
{ 
    \noindent\begin{tabular}{| p{2cm} | p{3.5cm} | p{3.5cm} | p{3.5cm} |}
        \hline
        \hline
        \textbf{Azonosító} & \textbf{Prioritás} & \textbf{Forrás} & \textbf{Ellenőrzés} \\
        \hline
        #1 & #2 & #3 & #4 \\
        \hline
    \end{tabular}
    \\[1ex]
    \textbf{Leírás:} #5 \\[0.2ex]
    \itshape}
{\\[5px]}

\newenvironment{use-case}[3]
{ 
    
    \vspace{10px}
    \noindent\begin{tabular}{| p{3cm} | p{12cm} |}
        \hline
        \hline
        \vspace{0.2ex} \textsc{\textbf{Use-case neve:}} & \vspace{0.2ex} \textsc{#1} \\[0.2ex]
        \hline
        \textbf{Rövid leírás:} & #2 \\
        \hline
        \textbf{Aktorok:} & #3 \\
        \hline
        \textbf{Forgatókönyv:} &  
    }
    {	\\
        \hline
\end{tabular} }

\newenvironment{fajllista}{
    \begin{longtable}{|p{4.5cm}|l|l| p{4.5cm} |}
        \hline
        \hline \multicolumn{1}{|c|}{\textbf{Fájl neve}} & \multicolumn{1}{c|}{\textbf{Méret}} & \multicolumn{1}{c|}{\textbf{Keletkezés ideje}} & \multicolumn{1}{c|}{\textbf{Tartalom}}  \\ \hline 
        \endfirsthead
        
        \multicolumn{3}{c}%
        {{\bfseries}} \\
        \hline \multicolumn{1}{|c|}{\textbf{Fájl neve}} & \multicolumn{1}{c|}{\textbf{Méret}} & \multicolumn{1}{c|}{\textbf{Keletkezés ideje}} & \multicolumn{1}{c|}{\textbf{Tartalom}}  \\ \hline 
        \endhead
        
    }{\end{longtable}}

\newcommand{\fajl}[4] {
    #1 & #2 & #3 & #4 \\ \hline
}

\newenvironment{ertekeles}{
    \begin{figure}[h]
        \centering
        \noindent\begin{tabular}{| p{5cm} | p{4cm} | p{4cm} |}
            \hline
            \hline
            \textbf{Tag neve} & \textbf{Tag neptun} & \textbf{Munka százalékban} \\	\hline
}{\end{tabular}\end{figure}}

\newcommand{\ertekelestag}[3]{ 
    #1 & #2 & #3 \\ \hline
}

\newcommand{\ertekelestagk}[2]{ 
    #1 & #2 \\ \hline
}

\newenvironment{ertekelesOra}{
    \begin{figure}[h]
        \centering
        \noindent\begin{tabular}{| p{5cm} | p{4cm} | p{4cm} |}
            \hline
            \hline
            \textbf{Tag neve} & \textbf{Tag neptun} & \textbf{Munka órában} \\	\hline
            
        }{\end{tabular}	\end{figure}}

\newenvironment{ertekelesKod}{
    \begin{figure}[h]
        \centering
        \noindent\begin{tabular}{| p{5cm} | p{4cm} |}
            \hline
            \hline
            \textbf{Fázis} & \textbf{Kódsorok száma} \\	\hline
            
        }{\end{tabular}	\end{figure}}

\newcommand{\test}[3]{
    \vspace{10px}
    \noindent\begin{tabular}{| p{3.5cm} | p{12cm} |}
        \hline
        \hline
        \vspace{0.2ex} \textsc{\textbf{Teszt-eset neve}} & \vspace{0.2ex} \textsc{#1} \\[0.2ex]
        \hline
        \textbf{Rövid leírás} & #2 \\
        \hline
        \textbf{Teszt célja} & #3 \\
        \hline
    \end{tabular}
}

\newenvironment{test-case}[4]
{ 
    \hspace*{0.4em} $\blacksquare $ \large Leírás   \\  \normalsize #1   \\[12px]
    \hspace*{0.4em} $\blacksquare $ \large Ellenőrzött funkcionalitás, várható hibahelyek  \\  \normalsize #2  \\[12px]
    \hspace*{0.4em} $\blacksquare $ \large Bemenet  \\  \normalsize #3  \\[12px]
    \hspace*{0.4em} $\blacksquare $ \large Elvárt kimenet \\  \normalsize #4 \\[12px]
}{}

\newenvironment{test-case-description}{\hspace*{0.4em} $\blacksquare $ \large Leírás \\[5px]  \normalsize}{\\[12px]}
\newenvironment{test-case-function}{\hspace*{0.4em} $\blacksquare $ \large Ellenőrzött funkcionalitás, várható hibahelyek \\[5px]  \normalsize}{\\[12px]}
\newenvironment{test-case-input}{\hspace*{0.4em}\parbox{\linewidth}{$\blacksquare $ \large Bemenet\vspace*{5px}}\normalsize}{\leavevmode\newline}
\newenvironment{test-case-output}{\hspace*{0.4em}\parbox{\linewidth}{$\blacksquare $ \large Elvárt kimenet\vspace{5px}}\normalsize}{\leavevmode\newline}

\newcommand{\testOK}[2]{
    \vspace{10px}
    \noindent\begin{tabular}{| p{4.2cm} | p{12cm} |}
        \hline
        \hline
        \textbf{Tesztelő neve} & \textsc{#1} \\[0.2ex]
        \hline
        \textbf{Teszt időpontja} & #2 \\
        \hline
    \end{tabular}
}

\newcommand{\testFAIL}[5]{
    \vspace{10px}
    \noindent\begin{tabular}{| p{4.2cm} | p{12cm} |}
        \hline
        \hline
        \textbf{Tesztelő neve} & \textsc{#1} \\[0.2ex]
        \hline
        \textbf{Teszt időpontja} & #2 \\
        \hline
        \textbf{Teszt eredménye} & #3 \\
        \hline
        \textbf{Lehetséges hibaokok} & #4 \\
        \hline
        \textbf{Változtatások} & #5 \\
        \hline
    \end{tabular}
}

\newenvironment{terv}{	
    \begin{longtable}{| l | p{8cm} | l |}
        \hline
        \hline
        \multicolumn{1}{|c|}{\textbf{Határidő}} & \multicolumn{1}{c|}{\textbf{Feladat}} & \multicolumn{1}{c|}{\textbf{Felelős}}  \\ \hline 
        \endfirsthead
        
        \multicolumn{3}{c}%
        {{\bfseries}} \\
        \hline \multicolumn{1}{|c|}{\textbf{Határidő}} & \multicolumn{1}{c|}{\textbf{Feladat}} & \multicolumn{1}{c|}{\textbf{Felelős}}  \\ \hline 
        \endhead
        
}{\end{longtable}}

\newcommand{\tervitem}[3]{ #1 & #2 & #3 \\ \hline}

\newenvironment{szotar}{
    \begin{longtable}{ p{4cm}  p{11cm} }
        
}{\end{longtable}}

\newcommand{\szotaritem}[2]{ \textbf{#1} & #2 \\[10px]}

\clearpage
A következő fejezetekben a $definitions$ fájlban található környezet és parancs definíciók kerülnek leírásra. Ahol beépített \LaTeX elemet említek külön jelzem.  

\section{Általános környezetek és parancsok}

Egyszerű képek beszúrására alkalmas a \textit{diagram} parancs, ami számozza a beszúrt képeket és mindig középre igazítja. 3 paraméterrel rendelkezik: 
\begin{enumerate}
	\item A kép elérési útja és neve, kiterjesztést nem kell megadni. Az elérésiút, mindig a fő dokumentum mappájából kiindulva adható meg.
	\item A képhez tartozó leírás megadására szolgál.
	\item A kép átméretezésére szolgáló paraméter: cm/px/in... mértékegységgel megadva.
\end{enumerate} 
Egy példaparancs és eredménye:
\begin{lstlisting}
\diagram{../docs/img/BMElogo}{Példa kép}{3cm}
\end{lstlisting}
\diagram{../docs/img/BMElogo}{Példa kép}{3cm}

A dokumentumban számozott definíciók és rövidítések megadására szolgálnak a következő környezetek, példaprogrammal:
\begin{lstlisting}
\begin{defi}
    Első definíció ...
\end{defi}
\begin{rov}
    Első rövidítés
\end{rov}
\end{lstlisting}
\begin{defi}
	Első definíció ...
\end{defi}
\begin{rov}
	Első rövidítés
\end{rov}


\section{Napló használata}
Minden egyes beadandónál le kell adni, az abban a szakaszban elkészített dokumentációkról és programról egy naplót. Ezt a $naplo$ környezet segítségével adhatjuk meg, a napló bejegyzéseit a $naplotag$ parancs szimbolizálja, mely 4 paramétert vár a napló fejlécének megfelelő sorredben, azaz a feladat elkezdésének időpontja, a feladat időtartama, a feladatban résztvevők, valamint leírás megadása. A leírás oszlopban, azaz az utolsó paraméterben lehetőségünk van a sorok tördelésére a $newline$ segítségével.
\begin{lstlisting}
\begin{naplo}
    \naplotag{feb. 18. 16h }{ 1 óra }{ Csapat }{ Értekezlet a tagok között \newline 
    Döntés: Segédeszközök kiválasztása (git, trello, drive)}
    \naplotag{feb. 19. 16h }{ 1.5 óra }{ Teszt János }{ Funkciók megtervezése}
\end{naplo}
\end{lstlisting}
    \begin{naplo}
        \naplotag{feb. 18. 16h }{ 1 óra }{ Csapat }{ Értekezlet a tagok között \newline Döntés: Segédeszközök kiválasztása (git, trello, drive)}
        \naplotag{feb. 19. 16h }{ 1.5 óra }{ Teszt János }{ Funkciók megtervezése}
    \end{naplo}

\section{Követelménydokumentálás}
Általános követelménynek leírására a $kovetelmeny$ környezet szolgál, megadási módja:
A környezet első paramétere a követelmény azonosítója, majd sorrendben a piorítás, forrás, ellenőrzés és leírás. Ezeket minden esetben meg kell adni paraméterként, azaz \{\} között. Opcionálisan a paraméterek után szabadon megadhatunk megjegyzést.
\begin{lstlisting}
\begin{kovetelmeny}
    {R31}
    {Alapvető}
    {Feladatkiadó}
    {bemutatás}
    {A játék futtatásához tetszőleges operációs rendszer alkalmas, amin elérhet...}
    Természetesen elvárás, hogy olyan illesztőprogramok telepítve legyenek, amik ...
\end{kovetelmeny}
\end{lstlisting}
\begin{kovetelmeny}
    %Azonosító
    {R31}
    %Prioritás
    {Alapvető}
    %Forrás
    {Feladatkiadó}
    %Ellenőrzés
    {bemutatás}
    %Leírás
    {A játék futtatásához tetszőleges operációs rendszer alkalmas, amin elérhető a Java Runtime Environment (JRE, legalább JRE 8-as verzió) , más szoftver nem szükséges.}
    Természetesen elvárás, hogy olyan illesztőprogramok telepítve legyenek, amik lehetővé teszik azon perifériák kezelését, amik a játékhoz kellenek (R33).
\end{kovetelmeny}

Funkcionális követelményekhez a $funkovetelmeny$ környezet való, használata részben megegyezik az előző környezettel. Rendre a következő paraméterek megadása szükséges: azonosító, pioritás, forrás, use-case neve, ellenőrzés, leírás és opcionálisan megjegyzés.

\begin{lstlisting}
\begin{funkovetelmeny}
    {R26}
    {Fontos}
    {Feladatkiírás}
    {Játék megtekintése, Jelzőrakéta építés}
    {Bemutatás}
    {Egy egységnyi munkával összeszerelhető és elsüthető a jelzőrakéta, ami...}
\end{funkovetelmeny}
\end{lstlisting}

\begin{funkovetelmeny}
    %Azonosító
    {R26}
    %Prioritás
    {Fontos}
    %Forrás
    {Feladatkiírás}
    %Use case
    {Játék megtekintése, Jelzőrakéta építés}
    %Ellenőrzés
    {Bemutatás}
    %Leírás
    {Egy egységnyi munkával összeszerelhető és elsüthető a jelzőrakéta, ami a játék megnyerését jelenti.}
\end{funkovetelmeny} 

\section{Használati esetek, use-case-ek megadása}
Használati eset leírására a $use-case$ környezet használható, mely sorrendben a következő paramétereket várja: használati eset neve, rövid leírása és a use-case-ben résztvevő aktorok. A környezet főtörzsébe a forgatóköny tetszőleges formázással megadható. A forgatókönyvnél az alternatívákat egy új sor beszúrásával, a korábban említett $newline$ paranccsal valamint kiemeléssel ($textbf$ parancs) szemléltetjük. 
\begin{lstlisting}
\begin{use-case}
    {Képesség használat}
    {A játékos a szereplő specifikus képességeit használhatja.}
    {Player}
    A játékos kijátssza a képességet. \newline
       \textbf{1.A.1} A sarkkutató játékos megnézheti, hogy egy adott szomszédos vagy a
       saját mezőjén hány játékos állhat. \newline
       \textbf{1.B.1} Az eszkimó játékos iglut építhet a kiválasztott mezőre.	
\end{use-case}
\end{lstlisting}
\begin{use-case}
    %név
    {Képesség használat}
    %rövid leírás
    {A játékos a szereplő specifikus képességeit használhatja.}
    %aktorok
    {Player}
    %forgatókönyv
    A játékos kijátssza a képességet. \newline
        \textbf{1.A.1} A sarkkutató játékos megnézheti, hogy egy adott szomszédos vagy a saját mezőjén hány játékos állhat. \newline
        \textbf{1.B.1} Az eszkimó játékos iglut építhet a kiválasztott mezőre.	
\end{use-case}

\section{Osztályok, interfészek dokumentálása}
Az osztály vagy interfész elemeinek megadására különböző környezeteket használhatunk: 
\begin{itemize}
    \item $class-template-responsibility$ Az osztály felelősségének leírására szolgáló környezet
    \item $class-template-baseclass$ Az ősosztályok felsorolására használható, az öröklés reprezentálása a $baseclass$ paranccsal történik.
    \item $class-template-interface$ A megvalósított interfészek felsorolására
    \item $class-template-attribute$ Az osztályban található attribútumok felsorolása és leírása. A felsoroláshoz a $classitem$ parancs használható, mely két paramétert vár: az adattag típusa, neve, megkötések... valamint a hozzá tartozó leírást
    \item $class-template-method$ Az osztályban megvalósuló függvények leírása, felsoroláshoz itt is a $classitem$ parancs kell.
    \item $class-template-statechart$ Ha szükséges, állapotgépek leírásához.
\end{itemize}
Egy példa néhány osztálykörnyezet és a $classitem$ parancs használatára
\begin{lstlisting}
\subsection*{DivingSuit}
\begin{class-template-responsibility}
    Absztrakt búvárruha osztály.
\end{class-template-responsibility}
\begin{class-template-baseclass}		
    Item \baseclass AutomaticItem
\end{class-template-baseclass}
\begin{class-template-method}
    \classitem{+fallInWater(p: Player): void}{Absztrakt, felülírható metódus, a ... }
\end{class-template-method}
\end{lstlisting}
\subsection*{DivingSuit}
\begin{class-template-responsibility}
    Absztrakt búvárruha osztály.
\end{class-template-responsibility}
\begin{class-template-baseclass}		
    Item \baseclass AutomaticItem
\end{class-template-baseclass}
\begin{class-template-method}
    \classitem{+fallInWater(p: Player): void}{Absztrakt, felülírható metódus, a játékos vízbe esésénél van szerepe.}
\end{class-template-method}

Lehetőség van felsorolás típus leírására is a $enum-template$ környezet segítségével, mely két paramétert vár: leírás és a literálok felsorolása.

\section{Tesztek leírása}
Tesztek összefoglaló leírására a $test$ parancs használható, mely a következő paramétereket várja: a teszt-eset neve, a teszt rövid leírása és a teszt célja.
\begin{lstlisting}
\test{A játék szimulálás}{A start automated parancs tesztesete}
{A parancs hatására a játék elindul, az egyes aktorok lépnek, léphetnek.}
\end{lstlisting}
\test{A játék szimulálás}{A start automated parancs tesztesete}{A parancs hatására a játék elindul, az egyes aktorok lépnek, léphetnek.}
\vspace{15px}

A tesztek részletes leírására a következő környezetek használhatók: 
\begin{itemize}
    \item $test-case-description$ a teszt leírása
    \item $test-case-function$ ellenőrzött funkciók, várható hibák
    \item $test-case-input$ a teszthez tartozó bement
    \item $test-case-output$ a teszt elvárt kimenete és eredménye
\end{itemize}
A teszt bemenetének és elvárt kimenetének megadására használandó a beépített $verbatim$ környezet, ami a szöveget mindenféle formázás nélkül, úgy jeleníti meg, ahogy beleírták (a tabokat nem jeleníti meg, helyette szóköz javasolt).

\begin{lstlisting}
\subsection*{A játékvilág létrehozása}
\begin{test-case-description}
    A teszteset célja az \texttt{init world} parancs lefutásának ellenőrzése.
\end{test-case-description}
\begin{test-case-function}
    A teszt alatt leellenőrizzük, hogy ...
\end{test-case-function}
\begin{test-case-input}
   \begin{verbatim}
        init world
        0: S 3
        ...
        print
    \end{verbatim}
\end{test-case-input}
\begin{test-case-output}
    A \texttt{print} parancs hatására a következő kell megjelenjen a kimeneten:
    \begin{verbatim}
        worlddata
        ...
        creatures 0
    \end{verbatim}
\end{test-case-output}
\end{lstlisting}

\subsection*{A játékvilág létrehozása}
\begin{test-case-description}
    A teszteset célja az \texttt{init world} parancs lefutásának ellenőrzése.
\end{test-case-description}
\begin{test-case-function}
    A teszt alatt leellenőrizzük, hogy ...
\end{test-case-function}
\begin{test-case-input}
    \begin{verbatim}
    init world
    0: S 3
    ...
    print
    \end{verbatim}
\end{test-case-input}
\begin{test-case-output}
    A \texttt{print} parancs hatására a következő kell megjelenjen a kimeneten:
    \begin{verbatim}
    worlddata
    ...
    creatures 0
    \end{verbatim}
\end{test-case-output}
A teszteket eredményük szempontjából egyszerűen két csoportba oszthatók: a teszt sikeresen lefutott, vagy valamin elhasalt (most lényegtelen, hogy ezt mi okozta). Előbbi esetén a $testOK$ parancs míg utóbbi esetén a $testFAIL$ parancs használandó. Helyes teszt esetén a két paraméter a tesztelő neve és a teszt időpontja, míg rossz teszteset esetén ez a teszt hibás eredményével, a hibaokok és a változtatások leírásával bővül a paraméterlista.
\begin{lstlisting}
\subsection*{Ásás törékeny ásóval}
\testFAIL{LaTex}{Ápr. 25. 13:50}{A törött ásó hátralevő használatai ...}
    {Akkor is csökkenti a BrokenShovel use-ja a use\_num-ját, ha már el van törve.}
    {Csak akkor csökken a use\_num, ha nagyobb, mint 0.}
\testOK{LaTex}{Ápr. 25. 14:00}
\end{lstlisting}
\subsection*{Ásás törékeny ásóval}
\testFAIL{LaTex}{Ápr. 25. 13:50}{A törött ásó hátralevő használatai negatívba megy, ha próbáljuk használni, miután eltörött.}{Akkor is csökkenti a BrokenShovel use-ja a use\_num-ját, ha már el van törve.}{Csak akkor csökken a use\_num, ha nagyobb, mint 0.}
\testOK{LaTex}{Ápr. 25. 14:00}

\section{Adminisztratív információk megjelenítése}
Ütemtervek vagy határidőnapló beszúrására a $terv$ környezet és a $tervitem$ parancs együtt használandó. A meglenített táblázatban a következők szerepelnek: feladat határideje, a feladat leírása és a szakasz felelőse, rendre ezen paramétereket várja a $tervitem$ parancs is:
\begin{lstlisting}
\begin{terv}
    \tervitem{ápr. 4}{Grafika 1. hf leadása}{ ?????? }
    \tervitem{ápr. 25}{Grafika 2. hf leadása}{ ?????? }
\end{terv}
\end{lstlisting}
\begin{terv}
    \tervitem{ápr. 4}{Grafika 1. hf leadása}{ ?????? }
    \tervitem{ápr. 25}{Grafika 2. hf leadása}{ ?????? }
\end{terv}
Szótárak, vagy kulcs-érték párok felsorolására a $szotar$ környezet és $szotaritem$ parancs együtt használandó. A parancs 2 paramétert vár: kulcs és érték.
\begin{lstlisting}
\begin{szotar}
    \szotaritem{Kulcs 1 }{Érték 1}
    \szotaritem{Hóásás }{Egy réteg hó eltakarítása, azaz törlése az adott...}
\end{szotar}
\end{lstlisting}
\begin{szotar}
    \szotaritem{Kulcs 1 }{Érték 1}
    \szotaritem{Hóásás }{Egy réteg hó eltakarítása, azaz törlése az adott jégtábláról. Egy egység munkába kerül.}
\end{szotar}
Fájlok beadása esetén a $fajllista$ környezet és $fajl$ parancs használandó. A parancsnak rendre a következő 4 paramétere van: fájl neve, mérete, keletkezés ideje, leírás.
\begin{lstlisting}
\begin{fajllista}
    \fajl{AutomaticItem.java}   {353 byte} {2020.03.26~21:05} {AutomaticItem osztály}
    \fajl{BasicDivingSuit.java} {797 byte} {2020.03.26~20:12} {BasicDivingSuit osztály}
\end{fajllista}
\end{lstlisting}
\begin{fajllista}
    \fajl{AutomaticItem.java}   {353 byte} {2020.03.26~21:05} {AutomaticItem osztály}
    \fajl{BasicDivingSuit.java} {797 byte} {2020.03.26~20:12} {BasicDivingSuit osztály}
\end{fajllista}
A szakaszonkénti teljesítményértékelések megadásához az $ertekeles$ környezet és $ertekelestag$ parancs használandó:
\begin{lstlisting}
\begin{ertekeles}
    \ertekelestag{LaTex Zoltán}{??????}{20\%}
    \ertekelestag{Git Attila}{??????}{20\%}
\end{ertekeles}
\end{lstlisting}
\begin{ertekeles}
    \ertekelestag{LaTex Zoltán}{??????}{20\%}
    \ertekelestag{Git Attila}{??????}{20\%}
\end{ertekeles}
A végső beadásnál az elvégzett munka összefoglalására az $ertekelesOra$ környezet használandó az előzőleg már alkalmazott $ertekelestag$ parancs felhasználásával:
\begin{lstlisting}
\begin{ertekelesOra}
    \ertekelestag{Teszt Tamás}{??????}{41}
    \ertekelestag{Java Dániel}{??????}{40}
\end{ertekelesOra}
\end{lstlisting}
\begin{ertekelesOra}
    \ertekelestag{Teszt Tamás}{??????}{41}
    \ertekelestag{Java Dániel}{??????}{40}
\end{ertekelesOra}
A végső beadásnál az beadott munkák összefoglalására az $ertekelesKod$ környezet használandó az $ertekelestagk$ parancs felhasználásával:
\begin{lstlisting}
\begin{ertekelesKod}
    \ertekelestagk{Szkeleton}{853}
    \ertekelestagk{Prototípus}{1357}
    \ertekelestagk{Grafikus változat}{2605}
    \ertekelestagk{Összesen}{4815}
\end{ertekelesKod}
\end{lstlisting}
\begin{ertekelesKod}
    \ertekelestagk{Szkeleton}{853}
    \ertekelestagk{Prototípus}{1357}
    \ertekelestagk{Grafikus változat}{2605}
    \ertekelestagk{Összesen}{4815}
\end{ertekelesKod}

\section{További hasznos parancsok}
\begin{itemize}
    \item \verb|\clearpage| egy új oldal megkezdése (beépített parancs)
    \item \verb|\urlref{url cím}{leírás}| hivatkozás beszúrása
    \item \verb|\newline|  új sor kezdése (beépített parancs)
    \item \verb|\textbf{szoveg}| félkövér szöveg (beépített parancs)
    \item \verb|\textit{szove}| dölt betűk (beépített parancs)
    \item \verb|\comment{szoveg}| piros, jobbra igazított szöveg beszúrása
    \item \verb|\setcounter{chapter}{-1}| A fejezetek számozásának átállítására (mindig a megadott számtól folytatódik a számozás, itt 0-tól) (beépített parancs)
    \item \verb|\setcounter{section}{-1}| A szakaszok számozásának átállítására (mindig a megadott számtól folytatódik a számozás, itt 0-tól) (beépített parancs)
    \item \verb|\tableofcontents| Tartalomjegyzék megjelenítése a végső beadásnál (beépített parancs)
    \item \verb|\input{fájl}| Egyes szakaszok nagyon nagy méreteket ölthetnek, így érdemes külön fájlba rakni, ennek behúzására alkalmas a parancs. (beépített parancs)
\end{itemize}

\section{Scriptek}
\subsection{Szótár generálás fájlból}
A $szotar.sh$ bash script alkalmas egy szöveges fájlból, amiben a kulcs-érték párok külön-külön sorban vannak és egy speciális karakterrel el vannak választva, abc-sorrendbe rakni és a már említett szintaktikát létrehozni a $szotar$ környezet számára. Vagyis a konzolon megjelent szöveget a \verb| \begin{szotar} \end{szotar}| közé kell beilleszteni, ugyanakkor érdemes külső fájlba rakni, és az $input$ parancsot használni. 

Alapértelmezetten a $\#$ karakter mentén történik a szétválasztás, de a 2. paraméterben megadhatunk eltérő karaktert is, így excel kompatibilis is lehet. Windows esetén PowerShell-el tudjuk a programot futtatni. Példák:
\begin{verbatim}
    ./szotar.sh szotar.txt
    ./szotar.sh szotar.txt "-"
    ./szotar.sh szotar.txt "|" > out.txt
\end{verbatim}
\subsection{Fájllista generálása}
Az aktuális mappában kiadva következő parancsot, a konzolra kiíródik a $fajllista$ környezetbe beillesztendő formázott lista: 
\begin{verbatim}
    stat -c "\fajl{%n}{%s byte}{%.16y}{...}" *
\end{verbatim}
Az output átirányítása fájlba:
\begin{verbatim}
    stat -c "\fajl{%n}{%s byte}{%.16y}{...}" * > fajllista.txt
\end{verbatim}

Sajnos egyes rendszereken a $POSIX$ szabvány miatt, nem a támogatja a fájlrendszer a létrehozási időpont tárolását. Ezért itt a $\%.16y$ az utólsó módosítást adja meg, ha a rendszer támogatja a létrehozási időpontot, akkor a $\%.16w$ paraméterrel lehetne lekérdezni.
\end{document}
